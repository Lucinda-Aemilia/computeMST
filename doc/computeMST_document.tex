\documentclass[oneside]{book}
\usepackage[bookmarks=true, colorlinks, linkcolor=gray]{hyperref}
\usepackage[top=1in, bottom=1in, left=1.25in, right=1.25in]{geometry}
\usepackage{cite}
\usepackage{graphicx}
\usepackage{listings}
\usepackage{color}

% Packages required by doxygen
\usepackage{fixltx2e}
\usepackage{calc}
\usepackage{doxygen}
\usepackage[export]{adjustbox} % also loads graphicx
\usepackage{graphicx}
\usepackage[utf8]{inputenc}
\usepackage{makeidx}
\usepackage{multicol}
\usepackage{multirow}
\PassOptionsToPackage{warn}{textcomp}
\usepackage{textcomp}
\usepackage[nointegrals]{wasysym}
\usepackage[table]{xcolor}


% \renewcommand\thesection{\arabic{section}}

\definecolor{mygreen}{rgb}{0,0.6,0}
\definecolor{mygray}{rgb}{0.5,0.5,0.5}
\definecolor{mymauve}{rgb}{0.58,0,0.82}

\lstset{ %
  backgroundcolor=\color{white},   % choose the background color; you must add \usepackage{color} or \usepackage{xcolor}
  basicstyle=\footnotesize\ttfamily,        % the size of the fonts that are used for the code
  breakatwhitespace=false,         % sets if automatic breaks should only happen at whitespace
  breaklines=true,                 % sets automatic line breaking
  captionpos=b,                    % sets the caption-position to bottom
  commentstyle=\color{mygreen}\ttfamily,    % comment style
  frame=single,	                   % adds a frame around the code
  keepspaces=true,                 % keeps spaces in text, useful for keeping indentation of code (possibly needs columns=flexible)
  keywordstyle=\color{blue},       % keyword style
  language=C++,                    % the language of the code
  rulecolor=\color{black},         % if not set, the frame-color may be changed on line-breaks within not-black text (e.g. comments (green here))
  showspaces=false,                % show spaces everywhere adding particular underscores; it overrides 'showstringspaces'
  showstringspaces=false,          % underline spaces within strings only
  showtabs=false,                  % show tabs within strings adding particular underscores
  stepnumber=2,                    % the step between two line-numbers. If it's 1, each line will be numbered
  stringstyle=\color{mymauve},     % string literal style
  tabsize=2,	                   % sets default tabsize to 2 spaces
}

\usepackage{caption}
\captionsetup{labelsep=space,justification=centering,font={bf},singlelinecheck=off,skip=4pt,position=top}

% Add search path
\makeatletter
\providecommand*\input@path{}
\newcommand*\addinputpath[1]{%
  \expandafter\def\expandafter\input@path
  \expandafter{\input@path#1}}%
\addinputpath{{doxygen/}}
\makeatother

\graphicspath{{doxygen/}{UML/}}

% �½ڲ���ҳ
\usepackage{xpatch}
\makeatletter
\xpatchcmd{\chapter}
  {\if@openright\cleardoublepage\else\clearpage\fi}{\par\relax}
  {}{}
\makeatother

\title{Individual Project 2\\ Constructing Minimum Spanning Trees\\ Software Design Document}
\author{Zhang Huimeng, 2015011280}

\begin{document}

\maketitle
\author
\clearpage

\tableofcontents
\clearpage


\part{Introduction}

    \chapter{Purpose}

        This software design document describes the architecture and system design of project ComputeMST. This is a homework of Fundamentals of Object-Oriented Programming class, Spring 2016. The homework requirement is as follows:

         (II) Project: On Constructing Minimum Spanning Trees (Difficulty: 1.1)

        Requirements and TIPS:

        (1) Implement the MST algorithm based on Voronoi diagram for computing Euclidean minimum spanning trees in 2D plane.

        (2) Please refer to \href{http://www.cs.princeton.edu/courses/archive/spr07/cos226/lectures/mst.pdf}{the following link} for MST construction algorithms. Either Prim's or Kruskal's algorithm is ok. Please refer to Page 39 for the idea of using Voronoi for MST computation.

        (3) Use the \href{http://www.cgal.org/}{CGAL Library} for constructing the Voronoi diagram and Delaunay triangulation.

        (4) Randomly generate 5 different testcases with more than 5000 points without duplicates to test the implemented method.

        (5) It is suggested that a validity checking function be implemented to verify the experimental results are correct. For example, you may directly apply Prim's or Kruskal's MST algorithm on the testcase to verify that the MST trees are correct.

        (6) Report the statistics of the experimental results, e.g., total runtime, total number of points, total length of the MST edges, etc. Figures and tables on the experimental results are welcome.

        (7) [This is not mandatory to finish, but it is a challenging topic] Again, can you compute the top K ($1 <= K <= 20$) minimum spanning trees?
    
    \chapter{Scope}

        I implemented the first six requirements of the homework, mainly using CGAL and Kruskal's MST algorithm. The program can load input from file or randomly generate input for itself; it has a GUI interface and can print results of the MST computation to file.

    \chapter{Definitions and Acronyms}
        
        


\part{System Overview}
    
    The program can compute Delaunay triangulation (via CGAL library) and MST for up to 10000 points in a 2-dimensional plane in 1 or 2 seconds. It can output the result to file. Also, it has a interface written with Freeglut.
    
\clearpage


\part{System Architecture}

    \chapter{Architectural Design} % ������ϵͳ����Ϊ����ģ�飬����ÿ��ģ��֮��Ĺ��ܺ��໥��ϵ

        The program can be divided into three parts: the Basics part, the Computational part and the Display part. 
        
        \begin{center}
        \includegraphics[width=\linewidth / 2]{main.png}
        \end{center}
        
        The basics part provides some useful tools for the Computational part, like statistics and timer.
        
        The computational part does the computation, and interacts with the Display part to provide a workable interface.
        
        The display part deals with the interface and user input.

    \chapter{Decomposition Description}
    
        \section{Basics}
        
            The \hyperlink{classcmst_1_1_stat}{Stat} class deals with statistics. It receives data in the form of floating-point number and can work continuously. It calculates the mean, maximum, minimum and standard deviation of the data.
        
            \begin{center}
            \includegraphics[width=\linewidth / 2]{classcmst_1_1_stat__coll__graph}
            \end{center}
            
            
        
        \section{Computational}
            
            The \hyperlink{classcmst_1_1_point2_d}{Point2D}, \hyperlink{classcmst_1_1_edge2_d}{Edge2D} and \hyperlink{classcmst_1_1_index_edge2_d}{IndexEdge2D} classes are the basic 2-dimensional computational geometry classes.

            \begin{center}
            \includegraphics[width=\linewidth / 2]{classcmst_1_1_edge2_d__inherit__graph}
            \end{center}

            \begin{center}
            \includegraphics[width=\linewidth / 2]{classcmst_1_1_point2_d__coll__graph}
            \end{center}
        
        \section{Display}

    \chapter{Design Rationale}

\clearpage


\part{Data Design}

    \chapter{Data Description} % data structure, data storage, process and organization

    \chapter{Data Dictionary} % list the objects and its attributes, methods and method parameters

\clearpage


\part{Human Interface Design}

    \chapter{Overview of Human Interface}

    \chapter{Screen Images}

    \chapter{Screen Objects and Actions}

\clearpage


\part{Design Patterns}

\clearpage


\part{Component Design}
%--- Begin generated contents ---
\chapter{Namespace Index}
\input{namespaces}
\chapter{Hierarchical Index}
\section{Class Hierarchy}
This inheritance list is sorted roughly, but not completely, alphabetically\+:\begin{DoxyCompactList}
\item \contentsline{section}{Delaunay}{\pageref{class_delaunay}}{}
\item \contentsline{section}{Edge}{\pageref{class_edge}}{}
\item \contentsline{section}{cmst\+:\+:Edge2D}{\pageref{classcmst_1_1_edge2_d}}{}
\begin{DoxyCompactList}
\item \contentsline{section}{cmst\+:\+:Index\+Edge2D}{\pageref{classcmst_1_1_index_edge2_d}}{}
\end{DoxyCompactList}
\item \contentsline{section}{cmst\+:\+:Graph2D}{\pageref{classcmst_1_1_graph2_d}}{}
\item \contentsline{section}{cmst\+:\+:Point2D}{\pageref{classcmst_1_1_point2_d}}{}
\item \contentsline{section}{cmst\+:\+:Stat}{\pageref{classcmst_1_1_stat}}{}
\item \contentsline{section}{cmst\+:\+:Window\+:\+:Test}{\pageref{structcmst_1_1_window_1_1_test}}{}
\item \contentsline{section}{cmst\+:\+:Timer}{\pageref{classcmst_1_1_timer}}{}
\item \contentsline{section}{Triangle}{\pageref{class_triangle}}{}
\item \contentsline{section}{Vec2f}{\pageref{class_vec2f}}{}
\item \contentsline{section}{cmst\+:\+:Window}{\pageref{classcmst_1_1_window}}{}
\end{DoxyCompactList}

\chapter{Class Index}
\section{Class List}
Here are the classes, structs, unions and interfaces with brief descriptions\+:\begin{DoxyCompactList}
\item\contentsline{section}{\hyperlink{class_delaunay}{Delaunay} }{\pageref{class_delaunay}}{}
\item\contentsline{section}{\hyperlink{class_edge}{Edge} }{\pageref{class_edge}}{}
\item\contentsline{section}{\hyperlink{classcmst_1_1_edge2_d}{cmst\+::\+Edge2D} }{\pageref{classcmst_1_1_edge2_d}}{}
\item\contentsline{section}{\hyperlink{classcmst_1_1_graph2_d}{cmst\+::\+Graph2D} }{\pageref{classcmst_1_1_graph2_d}}{}
\item\contentsline{section}{\hyperlink{classcmst_1_1_index_edge2_d}{cmst\+::\+Index\+Edge2D} }{\pageref{classcmst_1_1_index_edge2_d}}{}
\item\contentsline{section}{\hyperlink{classcmst_1_1_point2_d}{cmst\+::\+Point2D} }{\pageref{classcmst_1_1_point2_d}}{}
\item\contentsline{section}{\hyperlink{classcmst_1_1_stat}{cmst\+::\+Stat} }{\pageref{classcmst_1_1_stat}}{}
\item\contentsline{section}{\hyperlink{structcmst_1_1_window_1_1_test}{cmst\+::\+Window\+::\+Test} }{\pageref{structcmst_1_1_window_1_1_test}}{}
\item\contentsline{section}{\hyperlink{classcmst_1_1_timer}{cmst\+::\+Timer} }{\pageref{classcmst_1_1_timer}}{}
\item\contentsline{section}{\hyperlink{class_triangle}{Triangle} }{\pageref{class_triangle}}{}
\item\contentsline{section}{\hyperlink{class_vec2f}{Vec2f} }{\pageref{class_vec2f}}{}
\item\contentsline{section}{\hyperlink{classcmst_1_1_window}{cmst\+::\+Window} }{\pageref{classcmst_1_1_window}}{}
\end{DoxyCompactList}

\chapter{Namespace Documentation}
\hypertarget{namespacecmst}{}\section{cmst Namespace Reference}
\label{namespacecmst}\index{cmst@{cmst}}
\subsection*{Classes}
\begin{DoxyCompactItemize}
\item 
class \hyperlink{classcmst_1_1_edge2_d}{Edge2D}
\item 
class \hyperlink{classcmst_1_1_graph2_d}{Graph2D}
\item 
class \hyperlink{classcmst_1_1_index_edge2_d}{IndexEdge2D}
\item 
class \hyperlink{classcmst_1_1_point2_d}{Point2D}
\item 
class \hyperlink{classcmst_1_1_stat}{Stat}
\item 
class \hyperlink{classcmst_1_1_timer}{Timer}
\item 
class \hyperlink{classcmst_1_1_window}{Window}
\end{DoxyCompactItemize}
\subsection*{Enumerations}
\begin{DoxyCompactItemize}
\item 
enum \hyperlink{namespacecmst_a8dff7ccfde8a2770160b5f8dbf81c3b9}{Menu} \{ \\*
\hyperlink{namespacecmst_a8dff7ccfde8a2770160b5f8dbf81c3b9acc1b341b1550c4009942f0a14b433fb5}{NEW}, 
\hyperlink{namespacecmst_a8dff7ccfde8a2770160b5f8dbf81c3b9a5c1bbae108fea09abee0ae385bc729bf}{NEW\_4\_10}, 
\hyperlink{namespacecmst_a8dff7ccfde8a2770160b5f8dbf81c3b9a613b1d9e468c6e26e23c33a2abb2833e}{NEW\_11\_100}, 
\hyperlink{namespacecmst_a8dff7ccfde8a2770160b5f8dbf81c3b9a7ca958e9941ebe27acf2ae7511b735ba}{NEW\_101\_1000}, 
\\*
\hyperlink{namespacecmst_a8dff7ccfde8a2770160b5f8dbf81c3b9a35fe2b5c1c0015c3aee5049bee8b8acd}{NEW\_1001\_5000}, 
\hyperlink{namespacecmst_a8dff7ccfde8a2770160b5f8dbf81c3b9a73f732474a8fd1746e18e85ac18965f0}{NEW\_5001\_10000}, 
\hyperlink{namespacecmst_a8dff7ccfde8a2770160b5f8dbf81c3b9a4484f6b658f88772d77bfeaaa19b7a11}{SHOW}, 
\hyperlink{namespacecmst_a8dff7ccfde8a2770160b5f8dbf81c3b9acb33c06abf6f54d270a6a6f26b3b0ec4}{SHOW\_VORONOI}, 
\\*
\hyperlink{namespacecmst_a8dff7ccfde8a2770160b5f8dbf81c3b9a4a94d06c45900d709f4714d9fdebb3f0}{SHOW\_DELAUNAY}, 
\hyperlink{namespacecmst_a8dff7ccfde8a2770160b5f8dbf81c3b9a95c752e50b27aaae9aa758293828007f}{TEST}, 
\hyperlink{namespacecmst_a8dff7ccfde8a2770160b5f8dbf81c3b9a9996ad31ad035f42371059ad53098246}{TEST\_5}, 
\hyperlink{namespacecmst_a8dff7ccfde8a2770160b5f8dbf81c3b9a3f5dd18e08446227f6ce81fed096c339}{TEST\_20}, 
\\*
\hyperlink{namespacecmst_a8dff7ccfde8a2770160b5f8dbf81c3b9af46a16cac81f794824d0d87fb5588e33}{QUIT}
 \}\begin{DoxyCompactList}\small\item\em Return values for GLUT menus. \end{DoxyCompactList}
\end{DoxyCompactItemize}
\subsection*{Functions}
\begin{DoxyCompactItemize}
\item 
int \hyperlink{namespacecmst_a844037f018f3d5b7b1f1a5f4463da501}{randomInt} (int a, int b)
\item 
double \hyperlink{namespacecmst_a8df08a5847caeb65a6606968e40f336f}{randomDouble} (double a, double b)
\item 
std::vector$<$ \hyperlink{classcmst_1_1_point2_d}{Point2D} $>$ \hyperlink{namespacecmst_abd1822f67dc5d2be959508e628be0633}{TestcaseGenerator} (int num\_lower\_bound=100, int num\_upper\_bound=500, double x\_upper\_bound=MAX\_X, double y\_upper\_bound=MAX\_Y)
\end{DoxyCompactItemize}


\subsection{Enumeration Type Documentation}
\index{cmst@{cmst}!Menu@{Menu}}
\index{Menu@{Menu}!cmst@{cmst}}
\subsubsection[{\texorpdfstring{Menu}{Menu}}]{\setlength{\rightskip}{0pt plus 5cm}enum {\bf cmst::Menu}}\hypertarget{namespacecmst_a8dff7ccfde8a2770160b5f8dbf81c3b9}{}\label{namespacecmst_a8dff7ccfde8a2770160b5f8dbf81c3b9}


Return values for GLUT menus. 

\begin{Desc}
\item[Enumerator]\par
\begin{description}
\index{NEW@{NEW}!cmst@{cmst}}\index{cmst@{cmst}!NEW@{NEW}}\item[{\em 
NEW\hypertarget{namespacecmst_a8dff7ccfde8a2770160b5f8dbf81c3b9acc1b341b1550c4009942f0a14b433fb5}{}\label{namespacecmst_a8dff7ccfde8a2770160b5f8dbf81c3b9acc1b341b1550c4009942f0a14b433fb5}
}]\index{NEW\_4\_10@{NEW\_4\_10}!cmst@{cmst}}\index{cmst@{cmst}!NEW\_4\_10@{NEW\_4\_10}}\item[{\em 
NEW\_4\_10\hypertarget{namespacecmst_a8dff7ccfde8a2770160b5f8dbf81c3b9a5c1bbae108fea09abee0ae385bc729bf}{}\label{namespacecmst_a8dff7ccfde8a2770160b5f8dbf81c3b9a5c1bbae108fea09abee0ae385bc729bf}
}]\index{NEW\_11\_100@{NEW\_11\_100}!cmst@{cmst}}\index{cmst@{cmst}!NEW\_11\_100@{NEW\_11\_100}}\item[{\em 
NEW\_11\_100\hypertarget{namespacecmst_a8dff7ccfde8a2770160b5f8dbf81c3b9a613b1d9e468c6e26e23c33a2abb2833e}{}\label{namespacecmst_a8dff7ccfde8a2770160b5f8dbf81c3b9a613b1d9e468c6e26e23c33a2abb2833e}
}]\index{NEW\_101\_1000@{NEW\_101\_1000}!cmst@{cmst}}\index{cmst@{cmst}!NEW\_101\_1000@{NEW\_101\_1000}}\item[{\em 
NEW\_101\_1000\hypertarget{namespacecmst_a8dff7ccfde8a2770160b5f8dbf81c3b9a7ca958e9941ebe27acf2ae7511b735ba}{}\label{namespacecmst_a8dff7ccfde8a2770160b5f8dbf81c3b9a7ca958e9941ebe27acf2ae7511b735ba}
}]\index{NEW\_1001\_5000@{NEW\_1001\_5000}!cmst@{cmst}}\index{cmst@{cmst}!NEW\_1001\_5000@{NEW\_1001\_5000}}\item[{\em 
NEW\_1001\_5000\hypertarget{namespacecmst_a8dff7ccfde8a2770160b5f8dbf81c3b9a35fe2b5c1c0015c3aee5049bee8b8acd}{}\label{namespacecmst_a8dff7ccfde8a2770160b5f8dbf81c3b9a35fe2b5c1c0015c3aee5049bee8b8acd}
}]\index{NEW\_5001\_10000@{NEW\_5001\_10000}!cmst@{cmst}}\index{cmst@{cmst}!NEW\_5001\_10000@{NEW\_5001\_10000}}\item[{\em 
NEW\_5001\_10000\hypertarget{namespacecmst_a8dff7ccfde8a2770160b5f8dbf81c3b9a73f732474a8fd1746e18e85ac18965f0}{}\label{namespacecmst_a8dff7ccfde8a2770160b5f8dbf81c3b9a73f732474a8fd1746e18e85ac18965f0}
}]\index{SHOW@{SHOW}!cmst@{cmst}}\index{cmst@{cmst}!SHOW@{SHOW}}\item[{\em 
SHOW\hypertarget{namespacecmst_a8dff7ccfde8a2770160b5f8dbf81c3b9a4484f6b658f88772d77bfeaaa19b7a11}{}\label{namespacecmst_a8dff7ccfde8a2770160b5f8dbf81c3b9a4484f6b658f88772d77bfeaaa19b7a11}
}]\index{SHOW\_VORONOI@{SHOW\_VORONOI}!cmst@{cmst}}\index{cmst@{cmst}!SHOW\_VORONOI@{SHOW\_VORONOI}}\item[{\em 
SHOW\_VORONOI\hypertarget{namespacecmst_a8dff7ccfde8a2770160b5f8dbf81c3b9acb33c06abf6f54d270a6a6f26b3b0ec4}{}\label{namespacecmst_a8dff7ccfde8a2770160b5f8dbf81c3b9acb33c06abf6f54d270a6a6f26b3b0ec4}
}]\index{SHOW\_DELAUNAY@{SHOW\_DELAUNAY}!cmst@{cmst}}\index{cmst@{cmst}!SHOW\_DELAUNAY@{SHOW\_DELAUNAY}}\item[{\em 
SHOW\_DELAUNAY\hypertarget{namespacecmst_a8dff7ccfde8a2770160b5f8dbf81c3b9a4a94d06c45900d709f4714d9fdebb3f0}{}\label{namespacecmst_a8dff7ccfde8a2770160b5f8dbf81c3b9a4a94d06c45900d709f4714d9fdebb3f0}
}]\index{TEST@{TEST}!cmst@{cmst}}\index{cmst@{cmst}!TEST@{TEST}}\item[{\em 
TEST\hypertarget{namespacecmst_a8dff7ccfde8a2770160b5f8dbf81c3b9a95c752e50b27aaae9aa758293828007f}{}\label{namespacecmst_a8dff7ccfde8a2770160b5f8dbf81c3b9a95c752e50b27aaae9aa758293828007f}
}]\index{TEST\_5@{TEST\_5}!cmst@{cmst}}\index{cmst@{cmst}!TEST\_5@{TEST\_5}}\item[{\em 
TEST\_5\hypertarget{namespacecmst_a8dff7ccfde8a2770160b5f8dbf81c3b9a9996ad31ad035f42371059ad53098246}{}\label{namespacecmst_a8dff7ccfde8a2770160b5f8dbf81c3b9a9996ad31ad035f42371059ad53098246}
}]\index{TEST\_20@{TEST\_20}!cmst@{cmst}}\index{cmst@{cmst}!TEST\_20@{TEST\_20}}\item[{\em 
TEST\_20\hypertarget{namespacecmst_a8dff7ccfde8a2770160b5f8dbf81c3b9a3f5dd18e08446227f6ce81fed096c339}{}\label{namespacecmst_a8dff7ccfde8a2770160b5f8dbf81c3b9a3f5dd18e08446227f6ce81fed096c339}
}]\index{QUIT@{QUIT}!cmst@{cmst}}\index{cmst@{cmst}!QUIT@{QUIT}}\item[{\em 
QUIT\hypertarget{namespacecmst_a8dff7ccfde8a2770160b5f8dbf81c3b9af46a16cac81f794824d0d87fb5588e33}{}\label{namespacecmst_a8dff7ccfde8a2770160b5f8dbf81c3b9af46a16cac81f794824d0d87fb5588e33}
}]\end{description}
\end{Desc}


\subsection{Function Documentation}
\index{cmst@{cmst}!randomDouble@{randomDouble}}
\index{randomDouble@{randomDouble}!cmst@{cmst}}
\subsubsection[{\texorpdfstring{randomDouble(double a, double b)}{randomDouble(double a, double b)}}]{\setlength{\rightskip}{0pt plus 5cm}double cmst::randomDouble (
\begin{DoxyParamCaption}
\item[{double}]{a, }
\item[{double}]{b}
\end{DoxyParamCaption}
)}\hypertarget{namespacecmst_a8df08a5847caeb65a6606968e40f336f}{}\label{namespacecmst_a8df08a5847caeb65a6606968e40f336f}
Generate a floating-\/point number in the range \mbox{[}a, b\mbox{]}

Needs to be improved using other random classes 

Here is the caller graph for this function:
% FIG 0


\index{cmst@{cmst}!randomInt@{randomInt}}
\index{randomInt@{randomInt}!cmst@{cmst}}
\subsubsection[{\texorpdfstring{randomInt(int a, int b)}{randomInt(int a, int b)}}]{\setlength{\rightskip}{0pt plus 5cm}int cmst::randomInt (
\begin{DoxyParamCaption}
\item[{int}]{a, }
\item[{int}]{b}
\end{DoxyParamCaption}
)}\hypertarget{namespacecmst_a844037f018f3d5b7b1f1a5f4463da501}{}\label{namespacecmst_a844037f018f3d5b7b1f1a5f4463da501}
Generate an integer in the range \mbox{[}a, b\mbox{]}

Needs to be improved using other random classes 

Here is the caller graph for this function:
% FIG 1


\index{cmst@{cmst}!TestcaseGenerator@{TestcaseGenerator}}
\index{TestcaseGenerator@{TestcaseGenerator}!cmst@{cmst}}
\subsubsection[{\texorpdfstring{TestcaseGenerator(int num\_lower\_bound=100, int num\_upper\_bound=500, double x\_upper\_bound=MAX\_X, double y\_upper\_bound=MAX\_Y)}{TestcaseGenerator(int num_lower_bound=100, int num_upper_bound=500, double x_upper_bound=MAX_X, double y_upper_bound=MAX_Y)}}]{\setlength{\rightskip}{0pt plus 5cm}std::vector$<$ {\bf cmst::Point2D} $>$ cmst::TestcaseGenerator (
\begin{DoxyParamCaption}
\item[{int}]{num\_lower\_bound = {\ttfamily 100}, }
\item[{int}]{num\_upper\_bound = {\ttfamily 500}, }
\item[{double}]{x\_upper\_bound = {\ttfamily MAX\_X}, }
\item[{double}]{y\_upper\_bound = {\ttfamily MAX\_Y}}
\end{DoxyParamCaption}
)}\hypertarget{namespacecmst_abd1822f67dc5d2be959508e628be0633}{}\label{namespacecmst_abd1822f67dc5d2be959508e628be0633}
To-\/do: output the testcase file to ../testcase; Add erase and unique 

Here is the call graph for this function:
% FIG 2




Here is the caller graph for this function:
% FIG 3



\chapter{Class Documentation}
\hypertarget{class_delaunay}{}\section{Delaunay Class Reference}
\label{class_delaunay}\index{Delaunay@{Delaunay}}


{\ttfamily \#include $<$delaunay.h$>$}



Collaboration diagram for Delaunay:
% FIG 0
\subsection*{Public Member Functions}
\begin{DoxyCompactItemize}
\item 
std::vector$<$ \hyperlink{class_triangle}{Triangle} $>$ \hyperlink{class_delaunay_a97f07ff9e702dd4592fe353805a9f82c}{triangulate} (std::vector$<$ \hyperlink{class_vec2f}{Vec2f} $>$ \&vertices)
\item 
std::vector$<$ \hyperlink{class_triangle}{Triangle} $>$ \hyperlink{class_delaunay_a15d2799c1fa1b237e4f0afad610517c2}{getTriangles} ()
\item 
std::vector$<$ \hyperlink{class_edge}{Edge} $>$ \hyperlink{class_delaunay_a8ca53e3d46f367fc879de1905d8bbf3b}{getEdges} ()
\end{DoxyCompactItemize}
\subsection*{Private Attributes}
\begin{DoxyCompactItemize}
\item 
std::vector$<$ \hyperlink{class_triangle}{Triangle} $>$ \hyperlink{class_delaunay_a7f57c36e1eed644efe9e469c5212a826}{\_triangles}
\item 
std::vector$<$ \hyperlink{class_edge}{Edge} $>$ \hyperlink{class_delaunay_a6e53063d9dd4df3936c14165e64e482f}{\_edges}
\end{DoxyCompactItemize}


\subsection{Member Function Documentation}
\index{Delaunay@{Delaunay}!getEdges@{getEdges}}
\index{getEdges@{getEdges}!Delaunay@{Delaunay}}
\subsubsection[{\texorpdfstring{getEdges()}{getEdges()}}]{\setlength{\rightskip}{0pt plus 5cm}std::vector$<${\bf Edge}$>$ Delaunay::getEdges (
\begin{DoxyParamCaption}
{}
\end{DoxyParamCaption}
)\hspace{0.3cm}{\ttfamily [inline]}}\hypertarget{class_delaunay_a8ca53e3d46f367fc879de1905d8bbf3b}{}\label{class_delaunay_a8ca53e3d46f367fc879de1905d8bbf3b}
\index{Delaunay@{Delaunay}!getTriangles@{getTriangles}}
\index{getTriangles@{getTriangles}!Delaunay@{Delaunay}}
\subsubsection[{\texorpdfstring{getTriangles()}{getTriangles()}}]{\setlength{\rightskip}{0pt plus 5cm}std::vector$<${\bf Triangle}$>$ Delaunay::getTriangles (
\begin{DoxyParamCaption}
{}
\end{DoxyParamCaption}
)\hspace{0.3cm}{\ttfamily [inline]}}\hypertarget{class_delaunay_a15d2799c1fa1b237e4f0afad610517c2}{}\label{class_delaunay_a15d2799c1fa1b237e4f0afad610517c2}
\index{Delaunay@{Delaunay}!triangulate@{triangulate}}
\index{triangulate@{triangulate}!Delaunay@{Delaunay}}
\subsubsection[{\texorpdfstring{triangulate(std::vector$<$ Vec2f $>$ \&vertices)}{triangulate(std::vector< Vec2f > &vertices)}}]{\setlength{\rightskip}{0pt plus 5cm}std::vector$<$ {\bf Triangle} $>$ Delaunay::triangulate (
\begin{DoxyParamCaption}
\item[{std::vector$<$ {\bf Vec2f} $>$ \&}]{vertices}
\end{DoxyParamCaption}
)}\hypertarget{class_delaunay_a97f07ff9e702dd4592fe353805a9f82c}{}\label{class_delaunay_a97f07ff9e702dd4592fe353805a9f82c}


Here is the call graph for this function:
% FIG 1




\subsection{Member Data Documentation}
\index{Delaunay@{Delaunay}!\_edges@{\_edges}}
\index{\_edges@{\_edges}!Delaunay@{Delaunay}}
\subsubsection[{\texorpdfstring{\_edges}{_edges}}]{\setlength{\rightskip}{0pt plus 5cm}std::vector$<${\bf Edge}$>$ Delaunay::\_edges\hspace{0.3cm}{\ttfamily [private]}}\hypertarget{class_delaunay_a6e53063d9dd4df3936c14165e64e482f}{}\label{class_delaunay_a6e53063d9dd4df3936c14165e64e482f}
\index{Delaunay@{Delaunay}!\_triangles@{\_triangles}}
\index{\_triangles@{\_triangles}!Delaunay@{Delaunay}}
\subsubsection[{\texorpdfstring{\_triangles}{_triangles}}]{\setlength{\rightskip}{0pt plus 5cm}std::vector$<${\bf Triangle}$>$ Delaunay::\_triangles\hspace{0.3cm}{\ttfamily [private]}}\hypertarget{class_delaunay_a7f57c36e1eed644efe9e469c5212a826}{}\label{class_delaunay_a7f57c36e1eed644efe9e469c5212a826}

\hypertarget{class_edge}{}\section{Edge Class Reference}
\label{class_edge}\index{Edge@{Edge}}


{\ttfamily \#include $<$edge.h$>$}



Collaboration diagram for Edge:
% FIG 0
\subsection*{Public Member Functions}
\begin{DoxyCompactItemize}
\item 
\hyperlink{class_edge_a46424efbe9a3753f10ed69734968b42a}{Edge} (const \hyperlink{class_vec2f}{Vec2f} \&\hyperlink{class_edge_a52f5f3c82f0241458a80008649cf2669}{p1}, const \hyperlink{class_vec2f}{Vec2f} \&\hyperlink{class_edge_a1f8d9abda73daf83db1780825ae0be0c}{p2})
\item 
\hyperlink{class_edge_a70ed9f4d5f93c9cc5273a34361781532}{Edge} (const \hyperlink{class_edge}{Edge} \&e)
\item 
bool \hyperlink{class_edge_a23eeba05e956cd2a398280537f0c518f}{operator==} (const \hyperlink{class_edge}{Edge} \&e2) const 
\end{DoxyCompactItemize}
\subsection*{Public Attributes}
\begin{DoxyCompactItemize}
\item 
\hyperlink{class_vec2f}{Vec2f} \hyperlink{class_edge_a52f5f3c82f0241458a80008649cf2669}{p1}
\item 
\hyperlink{class_vec2f}{Vec2f} \hyperlink{class_edge_a1f8d9abda73daf83db1780825ae0be0c}{p2}
\end{DoxyCompactItemize}
\subsection*{Friends}
\begin{DoxyCompactItemize}
\item 
std::ostream \& \hyperlink{class_edge_a05e7e89dad7ec4c7ea81da0b1f487a78}{operator$<$$<$} (std::ostream \&str, const \hyperlink{class_edge}{Edge} \&e)
\end{DoxyCompactItemize}


\subsection{Constructor \& Destructor Documentation}
\index{Edge@{Edge}!Edge@{Edge}}
\index{Edge@{Edge}!Edge@{Edge}}
\subsubsection[{\texorpdfstring{Edge(const Vec2f \&p1, const Vec2f \&p2)}{Edge(const Vec2f &p1, const Vec2f &p2)}}]{\setlength{\rightskip}{0pt plus 5cm}Edge::Edge (
\begin{DoxyParamCaption}
\item[{const {\bf Vec2f} \&}]{p1, }
\item[{const {\bf Vec2f} \&}]{p2}
\end{DoxyParamCaption}
)\hspace{0.3cm}{\ttfamily [inline]}}\hypertarget{class_edge_a46424efbe9a3753f10ed69734968b42a}{}\label{class_edge_a46424efbe9a3753f10ed69734968b42a}
\index{Edge@{Edge}!Edge@{Edge}}
\index{Edge@{Edge}!Edge@{Edge}}
\subsubsection[{\texorpdfstring{Edge(const Edge \&e)}{Edge(const Edge &e)}}]{\setlength{\rightskip}{0pt plus 5cm}Edge::Edge (
\begin{DoxyParamCaption}
\item[{const {\bf Edge} \&}]{e}
\end{DoxyParamCaption}
)\hspace{0.3cm}{\ttfamily [inline]}}\hypertarget{class_edge_a70ed9f4d5f93c9cc5273a34361781532}{}\label{class_edge_a70ed9f4d5f93c9cc5273a34361781532}


\subsection{Member Function Documentation}
\index{Edge@{Edge}!operator==@{operator==}}
\index{operator==@{operator==}!Edge@{Edge}}
\subsubsection[{\texorpdfstring{operator==(const Edge \&e2) const }{operator==(const Edge &e2) const }}]{\setlength{\rightskip}{0pt plus 5cm}bool Edge::operator== (
\begin{DoxyParamCaption}
\item[{const {\bf Edge} \&}]{e2}
\end{DoxyParamCaption}
) const\hspace{0.3cm}{\ttfamily [inline]}}\hypertarget{class_edge_a23eeba05e956cd2a398280537f0c518f}{}\label{class_edge_a23eeba05e956cd2a398280537f0c518f}


\subsection{Friends And Related Function Documentation}
\index{Edge@{Edge}!operator$<$$<$@{operator$<$$<$}}
\index{operator$<$$<$@{operator$<$$<$}!Edge@{Edge}}
\subsubsection[{\texorpdfstring{operator$<$$<$}{operator<<}}]{\setlength{\rightskip}{0pt plus 5cm}std::ostream\& operator$<$$<$ (
\begin{DoxyParamCaption}
\item[{std::ostream \&}]{str, }
\item[{const {\bf Edge} \&}]{e}
\end{DoxyParamCaption}
)\hspace{0.3cm}{\ttfamily [friend]}}\hypertarget{class_edge_a05e7e89dad7ec4c7ea81da0b1f487a78}{}\label{class_edge_a05e7e89dad7ec4c7ea81da0b1f487a78}


\subsection{Member Data Documentation}
\index{Edge@{Edge}!p1@{p1}}
\index{p1@{p1}!Edge@{Edge}}
\subsubsection[{\texorpdfstring{p1}{p1}}]{\setlength{\rightskip}{0pt plus 5cm}{\bf Vec2f} Edge::p1}\hypertarget{class_edge_a52f5f3c82f0241458a80008649cf2669}{}\label{class_edge_a52f5f3c82f0241458a80008649cf2669}
\index{Edge@{Edge}!p2@{p2}}
\index{p2@{p2}!Edge@{Edge}}
\subsubsection[{\texorpdfstring{p2}{p2}}]{\setlength{\rightskip}{0pt plus 5cm}{\bf Vec2f} Edge::p2}\hypertarget{class_edge_a1f8d9abda73daf83db1780825ae0be0c}{}\label{class_edge_a1f8d9abda73daf83db1780825ae0be0c}

\hypertarget{classcmst_1_1_edge2_d}{}\section{cmst\+:\+:Edge2D Class Reference}
\label{classcmst_1_1_edge2_d}\index{cmst\+::\+Edge2D@{cmst\+::\+Edge2D}}


Inherited by \hyperlink{classcmst_1_1_index_edge2_d}{cmst\+::\+Index\+Edge2D}.

\subsection*{Public Member Functions}
\begin{DoxyCompactItemize}
\item 
{\bfseries Edge2D} (const \hyperlink{classcmst_1_1_point2_d}{Point2D} \&start, const \hyperlink{classcmst_1_1_point2_d}{Point2D} \&end)\hypertarget{classcmst_1_1_edge2_d_a0d1166315f84757395e889d3225e2ae0}{}\label{classcmst_1_1_edge2_d_a0d1166315f84757395e889d3225e2ae0}

\item 
double {\bfseries length} () const \hypertarget{classcmst_1_1_edge2_d_adaa859c8f6b412e1174abe8d8b429ce9}{}\label{classcmst_1_1_edge2_d_adaa859c8f6b412e1174abe8d8b429ce9}

\item 
\hyperlink{classcmst_1_1_point2_d}{Point2D} {\bfseries start} () const \hypertarget{classcmst_1_1_edge2_d_ad77218c63818fe92f43033ba1487ab89}{}\label{classcmst_1_1_edge2_d_ad77218c63818fe92f43033ba1487ab89}

\item 
\hyperlink{classcmst_1_1_point2_d}{Point2D} {\bfseries end} () const \hypertarget{classcmst_1_1_edge2_d_af02d43d8344759ac3709d318e26cdcee}{}\label{classcmst_1_1_edge2_d_af02d43d8344759ac3709d318e26cdcee}

\item 
bool {\bfseries operator$<$} (const \hyperlink{classcmst_1_1_edge2_d}{Edge2D} \&right)\hypertarget{classcmst_1_1_edge2_d_a58bd75e0a118f5192f5811cfacb6e7ac}{}\label{classcmst_1_1_edge2_d_a58bd75e0a118f5192f5811cfacb6e7ac}

\item 
bool {\bfseries operator==} (const \hyperlink{classcmst_1_1_edge2_d}{Edge2D} \&right)\hypertarget{classcmst_1_1_edge2_d_a1609ff699d629d691bf054acd2deb187}{}\label{classcmst_1_1_edge2_d_a1609ff699d629d691bf054acd2deb187}

\end{DoxyCompactItemize}
\subsection*{Protected Member Functions}
\begin{DoxyCompactItemize}
\item 
void {\bfseries swap\+\_\+points} ()\hypertarget{classcmst_1_1_edge2_d_aeb88dc66750f6c7967de0918e906abf4}{}\label{classcmst_1_1_edge2_d_aeb88dc66750f6c7967de0918e906abf4}

\end{DoxyCompactItemize}
\subsection*{Friends}
\begin{DoxyCompactItemize}
\item 
std\+::ostream \& {\bfseries operator$<$$<$} (std\+::ostream \&out, const \hyperlink{classcmst_1_1_edge2_d}{Edge2D} \&e)\hypertarget{classcmst_1_1_edge2_d_ae312c205375d240b1c8bc889f2c1c55e}{}\label{classcmst_1_1_edge2_d_ae312c205375d240b1c8bc889f2c1c55e}

\end{DoxyCompactItemize}


The documentation for this class was generated from the following file\+:\begin{DoxyCompactItemize}
\item 
Edge2\+D.\+h\end{DoxyCompactItemize}

\hypertarget{classcmst_1_1_graph2_d}{}\section{cmst\+:\+:Graph2D Class Reference}
\label{classcmst_1_1_graph2_d}\index{cmst\+::\+Graph2D@{cmst\+::\+Graph2D}}
\subsection*{Classes}
\begin{DoxyCompactItemize}
\item 
struct \hyperlink{structcmst_1_1_graph2_d_1_1_s_t}{ST}
\end{DoxyCompactItemize}
\subsection*{Public Member Functions}
\begin{DoxyCompactItemize}
\item 
\hyperlink{classcmst_1_1_graph2_d_a36cf583f9e2e59da2bed94c8569914d2}{Graph2D} (std\+::vector$<$ \hyperlink{classcmst_1_1_point2_d}{Point2D} $>$ \&points)
\item 
double \hyperlink{classcmst_1_1_graph2_d_a034d2d37b2d106c0e25d7ad7bc67907e}{Kruskal} ()
\item 
double {\bfseries naive\+Kruskal} ()\hypertarget{classcmst_1_1_graph2_d_af0db14845e80799be1d4fb15ca230110}{}\label{classcmst_1_1_graph2_d_af0db14845e80799be1d4fb15ca230110}

\item 
void {\bfseries draw\+Point} ()\hypertarget{classcmst_1_1_graph2_d_affec250ee22a067a28127b46ce976b90}{}\label{classcmst_1_1_graph2_d_affec250ee22a067a28127b46ce976b90}

\item 
void {\bfseries draw\+Delaunay} ()\hypertarget{classcmst_1_1_graph2_d_a2c4ed2ccd1fffc94c636929e531c4e3e}{}\label{classcmst_1_1_graph2_d_a2c4ed2ccd1fffc94c636929e531c4e3e}

\item 
void {\bfseries draw\+M\+ST} ()\hypertarget{classcmst_1_1_graph2_d_a96e388b819b351c8564eed9aecf58f7d}{}\label{classcmst_1_1_graph2_d_a96e388b819b351c8564eed9aecf58f7d}

\item 
bool {\bfseries print} (std\+::string file=\char`\"{}graph.\+txt\char`\"{})\hypertarget{classcmst_1_1_graph2_d_a0e0bafdd08a942ef01c66768b5021f09}{}\label{classcmst_1_1_graph2_d_a0e0bafdd08a942ef01c66768b5021f09}

\item 
void {\bfseries change\+S\+T\+Display} (int direc)\hypertarget{classcmst_1_1_graph2_d_abb6cf2245d6ac93f5553e28f8723fce5}{}\label{classcmst_1_1_graph2_d_abb6cf2245d6ac93f5553e28f8723fce5}

\item 
void {\bfseries print\+S\+T\+Info} ()\hypertarget{classcmst_1_1_graph2_d_a547a65e56068434928777eb7b4e59510}{}\label{classcmst_1_1_graph2_d_a547a65e56068434928777eb7b4e59510}

\item 
double {\bfseries mst\+Length} ()\hypertarget{classcmst_1_1_graph2_d_aea22c23fdbb3b9e91671562cb19730ed}{}\label{classcmst_1_1_graph2_d_aea22c23fdbb3b9e91671562cb19730ed}

\item 
int {\bfseries delaunay\+Time} () const \hypertarget{classcmst_1_1_graph2_d_a93a1d4d5d2dd08796e37bcba6de79341}{}\label{classcmst_1_1_graph2_d_a93a1d4d5d2dd08796e37bcba6de79341}

\item 
int {\bfseries mst\+Time} ()\hypertarget{classcmst_1_1_graph2_d_a3b596946f310f7024036d2c6a18985a3}{}\label{classcmst_1_1_graph2_d_a3b596946f310f7024036d2c6a18985a3}

\item 
int {\bfseries graph\+Construct\+Time} () const \hypertarget{classcmst_1_1_graph2_d_ad4756aa3f617493bd8b3f6ecfe099449}{}\label{classcmst_1_1_graph2_d_ad4756aa3f617493bd8b3f6ecfe099449}

\item 
int {\bfseries point\+Num} () const \hypertarget{classcmst_1_1_graph2_d_a0b18b38d5813b2fdbe8f5a8d6f92575d}{}\label{classcmst_1_1_graph2_d_a0b18b38d5813b2fdbe8f5a8d6f92575d}

\item 
int {\bfseries edge\+Num} () const \hypertarget{classcmst_1_1_graph2_d_ae2474e4dd9964cd18fc9926a296c82fd}{}\label{classcmst_1_1_graph2_d_ae2474e4dd9964cd18fc9926a296c82fd}

\item 
bool {\bfseries validate\+Done} () const \hypertarget{classcmst_1_1_graph2_d_ab7fbcf59b9ef4e9cc10015fd610cb4fc}{}\label{classcmst_1_1_graph2_d_ab7fbcf59b9ef4e9cc10015fd610cb4fc}

\end{DoxyCompactItemize}
\subsection*{Protected Member Functions}
\begin{DoxyCompactItemize}
\item 
int {\bfseries find\+Father} (int x)\hypertarget{classcmst_1_1_graph2_d_a0b860daa24f288eea5f490e12fcb67e2}{}\label{classcmst_1_1_graph2_d_a0b860daa24f288eea5f490e12fcb67e2}

\item 
void {\bfseries init\+Father} ()\hypertarget{classcmst_1_1_graph2_d_a5de76dfe02b4a13e0d3fe9a5e7ea7285}{}\label{classcmst_1_1_graph2_d_a5de76dfe02b4a13e0d3fe9a5e7ea7285}

\end{DoxyCompactItemize}
\subsection*{Protected Attributes}
\begin{DoxyCompactItemize}
\item 
std\+::vector$<$ int $>$ {\bfseries father}\hypertarget{classcmst_1_1_graph2_d_ad5d251f2f6f8b827af4404985fcec53c}{}\label{classcmst_1_1_graph2_d_ad5d251f2f6f8b827af4404985fcec53c}

\item 
std\+::vector$<$ \hyperlink{classcmst_1_1_point2_d}{Point2D} $>$ {\bfseries m\+\_\+points}\hypertarget{classcmst_1_1_graph2_d_a32456f3c630e34a56ce3109183142c10}{}\label{classcmst_1_1_graph2_d_a32456f3c630e34a56ce3109183142c10}

\item 
std\+::vector$<$ \hyperlink{classcmst_1_1_index_edge2_d}{Index\+Edge2D} $>$ {\bfseries m\+\_\+delaunay\+Edge}\hypertarget{classcmst_1_1_graph2_d_a6fe64b2078ec3c700a8a2e2bd77e2dae}{}\label{classcmst_1_1_graph2_d_a6fe64b2078ec3c700a8a2e2bd77e2dae}

\item 
std\+::vector$<$ \hyperlink{classcmst_1_1_index_edge2_d}{Index\+Edge2D} $>$ {\bfseries m\+\_\+\+M\+S\+T\+Edge}\hypertarget{classcmst_1_1_graph2_d_a1cc96b5251162964ac21f46955ac8271}{}\label{classcmst_1_1_graph2_d_a1cc96b5251162964ac21f46955ac8271}

\item 
std\+::vector$<$ \hyperlink{classcmst_1_1_index_edge2_d}{Index\+Edge2D} $>$ {\bfseries m\+\_\+edges}\hypertarget{classcmst_1_1_graph2_d_a31a6b042c1c1941ee59672b842c7d3c9}{}\label{classcmst_1_1_graph2_d_a31a6b042c1c1941ee59672b842c7d3c9}

\item 
std\+::vector$<$ C\+G\+A\+L\+::\+Object $>$ {\bfseries m\+\_\+voronoi\+Edge}\hypertarget{classcmst_1_1_graph2_d_a05e5ea6746bfd9d0ccd47308f4bbf1af}{}\label{classcmst_1_1_graph2_d_a05e5ea6746bfd9d0ccd47308f4bbf1af}

\item 
std\+::vector$<$ std\+::vector$<$ int $>$ $>$ {\bfseries m\+\_\+graph}\hypertarget{classcmst_1_1_graph2_d_a5df9c78edb4f5c68da11b01e44061dc5}{}\label{classcmst_1_1_graph2_d_a5df9c78edb4f5c68da11b01e44061dc5}

\item 
Delaunay {\bfseries m\+\_\+delaunay}\hypertarget{classcmst_1_1_graph2_d_af19557df59901e6078c2038652c95623}{}\label{classcmst_1_1_graph2_d_af19557df59901e6078c2038652c95623}

\item 
std\+::vector$<$ \hyperlink{structcmst_1_1_graph2_d_1_1_s_t}{ST} $>$ {\bfseries m\+\_\+\+ST}\hypertarget{classcmst_1_1_graph2_d_a829dc681f90679478b0ba9676af0bc03}{}\label{classcmst_1_1_graph2_d_a829dc681f90679478b0ba9676af0bc03}

\end{DoxyCompactItemize}


\subsection{Constructor \& Destructor Documentation}
\index{cmst\+::\+Graph2D@{cmst\+::\+Graph2D}!Graph2D@{Graph2D}}
\index{Graph2D@{Graph2D}!cmst\+::\+Graph2D@{cmst\+::\+Graph2D}}
\subsubsection[{\texorpdfstring{Graph2\+D(std\+::vector$<$ Point2\+D $>$ \&points)}{Graph2D(std::vector< Point2D > &points)}}]{\setlength{\rightskip}{0pt plus 5cm}cmst\+::\+Graph2\+D\+::\+Graph2D (
\begin{DoxyParamCaption}
\item[{std\+::vector$<$ {\bf Point2D} $>$ \&}]{points}
\end{DoxyParamCaption}
)}\hypertarget{classcmst_1_1_graph2_d_a36cf583f9e2e59da2bed94c8569914d2}{}\label{classcmst_1_1_graph2_d_a36cf583f9e2e59da2bed94c8569914d2}
Constructor which does everything.


\begin{DoxyItemize}
\item Compute naive\+Edge which contains all edges
\item Compute Delaunay graph 
\end{DoxyItemize}

\subsection{Member Function Documentation}
\index{cmst\+::\+Graph2D@{cmst\+::\+Graph2D}!Kruskal@{Kruskal}}
\index{Kruskal@{Kruskal}!cmst\+::\+Graph2D@{cmst\+::\+Graph2D}}
\subsubsection[{\texorpdfstring{Kruskal()}{Kruskal()}}]{\setlength{\rightskip}{0pt plus 5cm}double cmst\+::\+Graph2\+D\+::\+Kruskal (
\begin{DoxyParamCaption}
{}
\end{DoxyParamCaption}
)}\hypertarget{classcmst_1_1_graph2_d_a034d2d37b2d106c0e25d7ad7bc67907e}{}\label{classcmst_1_1_graph2_d_a034d2d37b2d106c0e25d7ad7bc67907e}
The Kruskal algorithm for finding the minimal spanning tree. \begin{DoxyReturn}{Returns}
The length of the M\+ST. 
\end{DoxyReturn}


The documentation for this class was generated from the following files\+:\begin{DoxyCompactItemize}
\item 
Graph2\+D.\+h\item 
Graph2\+D.\+cpp\end{DoxyCompactItemize}

\hypertarget{classcmst_1_1_index_edge2_d}{}\section{cmst\+:\+:Index\+Edge2D Class Reference}
\label{classcmst_1_1_index_edge2_d}\index{cmst\+::\+Index\+Edge2D@{cmst\+::\+Index\+Edge2D}}


Inherits \hyperlink{classcmst_1_1_edge2_d}{cmst\+::\+Edge2D}.

\subsection*{Public Member Functions}
\begin{DoxyCompactItemize}
\item 
{\bfseries Index\+Edge2D} (\hyperlink{classcmst_1_1_point2_d}{Point2D} p1, \hyperlink{classcmst_1_1_point2_d}{Point2D} p2, int index1, int index2)\hypertarget{classcmst_1_1_index_edge2_d_aa01ca9e529a319fffb5f500ab7218ef6}{}\label{classcmst_1_1_index_edge2_d_aa01ca9e529a319fffb5f500ab7218ef6}

\item 
int {\bfseries start\+Index} () const \hypertarget{classcmst_1_1_index_edge2_d_ab92c0b814ea9c354a71acfc84c871f72}{}\label{classcmst_1_1_index_edge2_d_ab92c0b814ea9c354a71acfc84c871f72}

\item 
int {\bfseries end\+Index} () const \hypertarget{classcmst_1_1_index_edge2_d_ad50c559d38e5857c03be255774deff12}{}\label{classcmst_1_1_index_edge2_d_ad50c559d38e5857c03be255774deff12}

\end{DoxyCompactItemize}
\subsection*{Friends}
\begin{DoxyCompactItemize}
\item 
std\+::ostream \& {\bfseries operator$<$$<$} (std\+::ostream \&str, const \hyperlink{classcmst_1_1_index_edge2_d}{Index\+Edge2D} \&e)\hypertarget{classcmst_1_1_index_edge2_d_a07a17f57bc3f2f18232965531f0b62b4}{}\label{classcmst_1_1_index_edge2_d_a07a17f57bc3f2f18232965531f0b62b4}

\end{DoxyCompactItemize}
\subsection*{Additional Inherited Members}


The documentation for this class was generated from the following file\+:\begin{DoxyCompactItemize}
\item 
Index\+Edge2\+D.\+h\end{DoxyCompactItemize}

\hypertarget{classcmst_1_1_point2_d}{}\section{cmst\+:\+:Point2D Class Reference}
\label{classcmst_1_1_point2_d}\index{cmst\+::\+Point2D@{cmst\+::\+Point2D}}
\subsection*{Public Member Functions}
\begin{DoxyCompactItemize}
\item 
{\bfseries Point2D} (double x=0.\+0, double y=0.\+0)\hypertarget{classcmst_1_1_point2_d_a2291fb012502a29399852810b5a081a8}{}\label{classcmst_1_1_point2_d_a2291fb012502a29399852810b5a081a8}

\item 
{\bfseries Point2D} (const \hyperlink{classcmst_1_1_point2_d}{Point2D} \&other)\hypertarget{classcmst_1_1_point2_d_a24e903416a709a44f844b93c7295ed67}{}\label{classcmst_1_1_point2_d_a24e903416a709a44f844b93c7295ed67}

\item 
double {\bfseries x} () const \hypertarget{classcmst_1_1_point2_d_a7745045ba529c4f2a2a0384974a42448}{}\label{classcmst_1_1_point2_d_a7745045ba529c4f2a2a0384974a42448}

\item 
double {\bfseries y} () const \hypertarget{classcmst_1_1_point2_d_a15a4383f1c181b7c7518ccac6f578564}{}\label{classcmst_1_1_point2_d_a15a4383f1c181b7c7518ccac6f578564}

\item 
bool {\bfseries operator$<$} (const \hyperlink{classcmst_1_1_point2_d}{Point2D} \&right) const \hypertarget{classcmst_1_1_point2_d_ada9efa4e0f4d7906e2ccaf2afeefea38}{}\label{classcmst_1_1_point2_d_ada9efa4e0f4d7906e2ccaf2afeefea38}

\item 
bool {\bfseries operator==} (const \hyperlink{classcmst_1_1_point2_d}{Point2D} \&right) const \hypertarget{classcmst_1_1_point2_d_a2181e34aa07d5ce5ac48545408924d22}{}\label{classcmst_1_1_point2_d_a2181e34aa07d5ce5ac48545408924d22}

\end{DoxyCompactItemize}
\subsection*{Friends}
\begin{DoxyCompactItemize}
\item 
std\+::ostream \& {\bfseries operator$<$$<$} (std\+::ostream \&out, const \hyperlink{classcmst_1_1_point2_d}{Point2D} \&p)\hypertarget{classcmst_1_1_point2_d_a538910ff384fb4f3f02d785a99e222cf}{}\label{classcmst_1_1_point2_d_a538910ff384fb4f3f02d785a99e222cf}

\end{DoxyCompactItemize}


The documentation for this class was generated from the following file\+:\begin{DoxyCompactItemize}
\item 
Point2\+D.\+h\end{DoxyCompactItemize}

\hypertarget{classcmst_1_1_stat}{}\section{cmst\+:\+:Stat Class Reference}
\label{classcmst_1_1_stat}\index{cmst\+::\+Stat@{cmst\+::\+Stat}}
\subsection*{Public Member Functions}
\begin{DoxyCompactItemize}
\item 
void {\bfseries record} (double data)\hypertarget{classcmst_1_1_stat_ab1e2fe7c367da505a6b5f1fb5eb619d2}{}\label{classcmst_1_1_stat_ab1e2fe7c367da505a6b5f1fb5eb619d2}

\item 
double {\bfseries min} () const \hypertarget{classcmst_1_1_stat_a1a6a92dee526145fb289d79a94afe3ae}{}\label{classcmst_1_1_stat_a1a6a92dee526145fb289d79a94afe3ae}

\item 
double {\bfseries max} () const \hypertarget{classcmst_1_1_stat_ab42898f6611aa9f1ad028f99c3d4242a}{}\label{classcmst_1_1_stat_ab42898f6611aa9f1ad028f99c3d4242a}

\item 
int {\bfseries count} () const \hypertarget{classcmst_1_1_stat_ab8c6707fa4739fda8b27a2481df25c35}{}\label{classcmst_1_1_stat_ab8c6707fa4739fda8b27a2481df25c35}

\item 
double {\bfseries mean} ()\hypertarget{classcmst_1_1_stat_aa40d8d516e7f866146d91866d63faf2b}{}\label{classcmst_1_1_stat_aa40d8d516e7f866146d91866d63faf2b}

\item 
double {\bfseries standard\+Deviation} ()\hypertarget{classcmst_1_1_stat_abfbaefc3a4174643a2eb282251fd86a5}{}\label{classcmst_1_1_stat_abfbaefc3a4174643a2eb282251fd86a5}

\item 
std\+::string {\bfseries print} ()\hypertarget{classcmst_1_1_stat_a03d1a0f52e2ea72cfab11a426726aea4}{}\label{classcmst_1_1_stat_a03d1a0f52e2ea72cfab11a426726aea4}

\end{DoxyCompactItemize}


The documentation for this class was generated from the following file\+:\begin{DoxyCompactItemize}
\item 
Stat.\+h\end{DoxyCompactItemize}

\hypertarget{structcmst_1_1_window_1_1_test}{}\section{cmst\+:\+:Window\+:\+:Test Struct Reference}
\label{structcmst_1_1_window_1_1_test}\index{cmst\+::\+Window\+::\+Test@{cmst\+::\+Window\+::\+Test}}
\subsection*{Public Attributes}
\begin{DoxyCompactItemize}
\item 
bool {\bfseries m\+\_\+display\+Test}\hypertarget{structcmst_1_1_window_1_1_test_a6e01140b018e7c479b15ae6499d8b9e8}{}\label{structcmst_1_1_window_1_1_test_a6e01140b018e7c479b15ae6499d8b9e8}

\item 
int {\bfseries m\+\_\+display\+Test\+Num}\hypertarget{structcmst_1_1_window_1_1_test_a0968826c727c33df28c160c4095e0b2f}{}\label{structcmst_1_1_window_1_1_test_a0968826c727c33df28c160c4095e0b2f}

\item 
std\+::vector$<$ \hyperlink{classcmst_1_1_graph2_d}{Graph2D} $>$ {\bfseries m\+\_\+test\+Graphs}\hypertarget{structcmst_1_1_window_1_1_test_a42f44814c524399a3c1f547a64218c1a}{}\label{structcmst_1_1_window_1_1_test_a42f44814c524399a3c1f547a64218c1a}

\item 
\hyperlink{classcmst_1_1_stat}{Stat} {\bfseries m\+\_\+delaunay\+Time\+Stat}\hypertarget{structcmst_1_1_window_1_1_test_a6f0ad01079d1c5b149282e31db771772}{}\label{structcmst_1_1_window_1_1_test_a6f0ad01079d1c5b149282e31db771772}

\item 
\hyperlink{classcmst_1_1_stat}{Stat} {\bfseries m\+\_\+graph\+Construct\+Time\+Stat}\hypertarget{structcmst_1_1_window_1_1_test_a4ff72191889ff43d7d80517120a97fae}{}\label{structcmst_1_1_window_1_1_test_a4ff72191889ff43d7d80517120a97fae}

\item 
\hyperlink{classcmst_1_1_stat}{Stat} {\bfseries m\+\_\+mst\+Time\+Stat}\hypertarget{structcmst_1_1_window_1_1_test_a103594f227a761c33c06b8dd68713c18}{}\label{structcmst_1_1_window_1_1_test_a103594f227a761c33c06b8dd68713c18}

\end{DoxyCompactItemize}


The documentation for this struct was generated from the following file\+:\begin{DoxyCompactItemize}
\item 
Window.\+h\end{DoxyCompactItemize}

\hypertarget{classcmst_1_1_timer}{}\section{cmst\+:\+:Timer Class Reference}
\label{classcmst_1_1_timer}\index{cmst\+::\+Timer@{cmst\+::\+Timer}}


{\ttfamily \#include $<$Timer.\+h$>$}



Collaboration diagram for cmst\+:\+:Timer\+:
% FIG 0
\subsection*{Public Member Functions}
\begin{DoxyCompactItemize}
\item 
\hyperlink{classcmst_1_1_timer_a92455552cedec79b452d3a0f89a0e32e}{Timer} ()
\item 
int \hyperlink{classcmst_1_1_timer_af0145067ee61560f2363dc4a1cb552b9}{time} ()
\item 
void \hyperlink{classcmst_1_1_timer_a8c7011cc646563211b681249df6dd4cf}{reset} ()
\end{DoxyCompactItemize}
\subsection*{Private Attributes}
\begin{DoxyCompactItemize}
\item 
int \hyperlink{classcmst_1_1_timer_a0fa8671c0b1dc3efca0f4dc5dfe98fc2}{m\+\_\+begin}
\end{DoxyCompactItemize}


\subsection{Constructor \& Destructor Documentation}
\index{cmst\+::\+Timer@{cmst\+::\+Timer}!Timer@{Timer}}
\index{Timer@{Timer}!cmst\+::\+Timer@{cmst\+::\+Timer}}
\subsubsection[{\texorpdfstring{Timer()}{Timer()}}]{\setlength{\rightskip}{0pt plus 5cm}cmst\+::\+Timer\+::\+Timer (
\begin{DoxyParamCaption}
{}
\end{DoxyParamCaption}
)\hspace{0.3cm}{\ttfamily [inline]}}\hypertarget{classcmst_1_1_timer_a92455552cedec79b452d3a0f89a0e32e}{}\label{classcmst_1_1_timer_a92455552cedec79b452d3a0f89a0e32e}


\subsection{Member Function Documentation}
\index{cmst\+::\+Timer@{cmst\+::\+Timer}!reset@{reset}}
\index{reset@{reset}!cmst\+::\+Timer@{cmst\+::\+Timer}}
\subsubsection[{\texorpdfstring{reset()}{reset()}}]{\setlength{\rightskip}{0pt plus 5cm}void cmst\+::\+Timer\+::reset (
\begin{DoxyParamCaption}
{}
\end{DoxyParamCaption}
)\hspace{0.3cm}{\ttfamily [inline]}}\hypertarget{classcmst_1_1_timer_a8c7011cc646563211b681249df6dd4cf}{}\label{classcmst_1_1_timer_a8c7011cc646563211b681249df6dd4cf}


Here is the caller graph for this function\+:
% FIG 1


\index{cmst\+::\+Timer@{cmst\+::\+Timer}!time@{time}}
\index{time@{time}!cmst\+::\+Timer@{cmst\+::\+Timer}}
\subsubsection[{\texorpdfstring{time()}{time()}}]{\setlength{\rightskip}{0pt plus 5cm}int cmst\+::\+Timer\+::time (
\begin{DoxyParamCaption}
{}
\end{DoxyParamCaption}
)\hspace{0.3cm}{\ttfamily [inline]}}\hypertarget{classcmst_1_1_timer_af0145067ee61560f2363dc4a1cb552b9}{}\label{classcmst_1_1_timer_af0145067ee61560f2363dc4a1cb552b9}


Here is the caller graph for this function\+:
% FIG 2




\subsection{Member Data Documentation}
\index{cmst\+::\+Timer@{cmst\+::\+Timer}!m\+\_\+begin@{m\+\_\+begin}}
\index{m\+\_\+begin@{m\+\_\+begin}!cmst\+::\+Timer@{cmst\+::\+Timer}}
\subsubsection[{\texorpdfstring{m\+\_\+begin}{m_begin}}]{\setlength{\rightskip}{0pt plus 5cm}int cmst\+::\+Timer\+::m\+\_\+begin\hspace{0.3cm}{\ttfamily [private]}}\hypertarget{classcmst_1_1_timer_a0fa8671c0b1dc3efca0f4dc5dfe98fc2}{}\label{classcmst_1_1_timer_a0fa8671c0b1dc3efca0f4dc5dfe98fc2}

\hypertarget{class_triangle}{}\section{Triangle Class Reference}
\label{class_triangle}\index{Triangle@{Triangle}}


{\ttfamily \#include $<$triangle.h$>$}



Collaboration diagram for Triangle:
% FIG 0
\subsection*{Public Member Functions}
\begin{DoxyCompactItemize}
\item 
\hyperlink{class_triangle_ac91f2225feea061bdc8b90b5ad05ab42}{Triangle} (const \hyperlink{class_vec2f}{Vec2f} \&\_p1, const \hyperlink{class_vec2f}{Vec2f} \&\_p2, const \hyperlink{class_vec2f}{Vec2f} \&\_p3)
\item 
bool \hyperlink{class_triangle_a8b279a0dff36453b04b359dc1c3dfb50}{containsVertex} (const \hyperlink{class_vec2f}{Vec2f} \&v)
\item 
bool \hyperlink{class_triangle_a7a46dcbaa3532abb09c561ae79a3cc77}{circumCircleContains} (const \hyperlink{class_vec2f}{Vec2f} \&v)
\item 
bool \hyperlink{class_triangle_a12315ef4973d7f56ca51760330391e05}{operator==} (const \hyperlink{class_triangle}{Triangle} \&t2) const 
\end{DoxyCompactItemize}
\subsection*{Public Attributes}
\begin{DoxyCompactItemize}
\item 
\hyperlink{class_vec2f}{Vec2f} \hyperlink{class_triangle_af2ff757f1bff3177e1a0e8c19664edbc}{p1}
\item 
\hyperlink{class_vec2f}{Vec2f} \hyperlink{class_triangle_a6a7d000238fb26c47181de1e7a502037}{p2}
\item 
\hyperlink{class_vec2f}{Vec2f} \hyperlink{class_triangle_a43b6750934c84aff2f34b38d2f26d012}{p3}
\item 
\hyperlink{class_edge}{Edge} \hyperlink{class_triangle_ab91aa48d8a4c25deecd1a35433a38260}{e1}
\item 
\hyperlink{class_edge}{Edge} \hyperlink{class_triangle_a5ad4339f6fb03adb4f6ee840ace34d82}{e2}
\item 
\hyperlink{class_edge}{Edge} \hyperlink{class_triangle_aa55fc8f8662cba36b227858760bb173c}{e3}
\end{DoxyCompactItemize}
\subsection*{Friends}
\begin{DoxyCompactItemize}
\item 
std::ostream \& \hyperlink{class_triangle_a865c96e9af49664ea93f369217f1f4f4}{operator$<$$<$} (std::ostream \&str, const \hyperlink{class_triangle}{Triangle} \&t)
\end{DoxyCompactItemize}


\subsection{Constructor \& Destructor Documentation}
\index{Triangle@{Triangle}!Triangle@{Triangle}}
\index{Triangle@{Triangle}!Triangle@{Triangle}}
\subsubsection[{\texorpdfstring{Triangle(const Vec2f \&\_p1, const Vec2f \&\_p2, const Vec2f \&\_p3)}{Triangle(const Vec2f &_p1, const Vec2f &_p2, const Vec2f &_p3)}}]{\setlength{\rightskip}{0pt plus 5cm}Triangle::Triangle (
\begin{DoxyParamCaption}
\item[{const {\bf Vec2f} \&}]{\_p1, }
\item[{const {\bf Vec2f} \&}]{\_p2, }
\item[{const {\bf Vec2f} \&}]{\_p3}
\end{DoxyParamCaption}
)}\hypertarget{class_triangle_ac91f2225feea061bdc8b90b5ad05ab42}{}\label{class_triangle_ac91f2225feea061bdc8b90b5ad05ab42}


\subsection{Member Function Documentation}
\index{Triangle@{Triangle}!circumCircleContains@{circumCircleContains}}
\index{circumCircleContains@{circumCircleContains}!Triangle@{Triangle}}
\subsubsection[{\texorpdfstring{circumCircleContains(const Vec2f \&v)}{circumCircleContains(const Vec2f &v)}}]{\setlength{\rightskip}{0pt plus 5cm}bool Triangle::circumCircleContains (
\begin{DoxyParamCaption}
\item[{const {\bf Vec2f} \&}]{v}
\end{DoxyParamCaption}
)}\hypertarget{class_triangle_a7a46dcbaa3532abb09c561ae79a3cc77}{}\label{class_triangle_a7a46dcbaa3532abb09c561ae79a3cc77}
\index{Triangle@{Triangle}!containsVertex@{containsVertex}}
\index{containsVertex@{containsVertex}!Triangle@{Triangle}}
\subsubsection[{\texorpdfstring{containsVertex(const Vec2f \&v)}{containsVertex(const Vec2f &v)}}]{\setlength{\rightskip}{0pt plus 5cm}bool Triangle::containsVertex (
\begin{DoxyParamCaption}
\item[{const {\bf Vec2f} \&}]{v}
\end{DoxyParamCaption}
)}\hypertarget{class_triangle_a8b279a0dff36453b04b359dc1c3dfb50}{}\label{class_triangle_a8b279a0dff36453b04b359dc1c3dfb50}


Here is the caller graph for this function:
% FIG 1


\index{Triangle@{Triangle}!operator==@{operator==}}
\index{operator==@{operator==}!Triangle@{Triangle}}
\subsubsection[{\texorpdfstring{operator==(const Triangle \&t2) const }{operator==(const Triangle &t2) const }}]{\setlength{\rightskip}{0pt plus 5cm}bool Triangle::operator== (
\begin{DoxyParamCaption}
\item[{const {\bf Triangle} \&}]{t2}
\end{DoxyParamCaption}
) const\hspace{0.3cm}{\ttfamily [inline]}}\hypertarget{class_triangle_a12315ef4973d7f56ca51760330391e05}{}\label{class_triangle_a12315ef4973d7f56ca51760330391e05}


\subsection{Friends And Related Function Documentation}
\index{Triangle@{Triangle}!operator$<$$<$@{operator$<$$<$}}
\index{operator$<$$<$@{operator$<$$<$}!Triangle@{Triangle}}
\subsubsection[{\texorpdfstring{operator$<$$<$}{operator<<}}]{\setlength{\rightskip}{0pt plus 5cm}std::ostream\& operator$<$$<$ (
\begin{DoxyParamCaption}
\item[{std::ostream \&}]{str, }
\item[{const {\bf Triangle} \&}]{t}
\end{DoxyParamCaption}
)\hspace{0.3cm}{\ttfamily [friend]}}\hypertarget{class_triangle_a865c96e9af49664ea93f369217f1f4f4}{}\label{class_triangle_a865c96e9af49664ea93f369217f1f4f4}
This has to be a friend. 

\subsection{Member Data Documentation}
\index{Triangle@{Triangle}!e1@{e1}}
\index{e1@{e1}!Triangle@{Triangle}}
\subsubsection[{\texorpdfstring{e1}{e1}}]{\setlength{\rightskip}{0pt plus 5cm}{\bf Edge} Triangle::e1}\hypertarget{class_triangle_ab91aa48d8a4c25deecd1a35433a38260}{}\label{class_triangle_ab91aa48d8a4c25deecd1a35433a38260}
\index{Triangle@{Triangle}!e2@{e2}}
\index{e2@{e2}!Triangle@{Triangle}}
\subsubsection[{\texorpdfstring{e2}{e2}}]{\setlength{\rightskip}{0pt plus 5cm}{\bf Edge} Triangle::e2}\hypertarget{class_triangle_a5ad4339f6fb03adb4f6ee840ace34d82}{}\label{class_triangle_a5ad4339f6fb03adb4f6ee840ace34d82}
\index{Triangle@{Triangle}!e3@{e3}}
\index{e3@{e3}!Triangle@{Triangle}}
\subsubsection[{\texorpdfstring{e3}{e3}}]{\setlength{\rightskip}{0pt plus 5cm}{\bf Edge} Triangle::e3}\hypertarget{class_triangle_aa55fc8f8662cba36b227858760bb173c}{}\label{class_triangle_aa55fc8f8662cba36b227858760bb173c}
\index{Triangle@{Triangle}!p1@{p1}}
\index{p1@{p1}!Triangle@{Triangle}}
\subsubsection[{\texorpdfstring{p1}{p1}}]{\setlength{\rightskip}{0pt plus 5cm}{\bf Vec2f} Triangle::p1}\hypertarget{class_triangle_af2ff757f1bff3177e1a0e8c19664edbc}{}\label{class_triangle_af2ff757f1bff3177e1a0e8c19664edbc}
\index{Triangle@{Triangle}!p2@{p2}}
\index{p2@{p2}!Triangle@{Triangle}}
\subsubsection[{\texorpdfstring{p2}{p2}}]{\setlength{\rightskip}{0pt plus 5cm}{\bf Vec2f} Triangle::p2}\hypertarget{class_triangle_a6a7d000238fb26c47181de1e7a502037}{}\label{class_triangle_a6a7d000238fb26c47181de1e7a502037}
\index{Triangle@{Triangle}!p3@{p3}}
\index{p3@{p3}!Triangle@{Triangle}}
\subsubsection[{\texorpdfstring{p3}{p3}}]{\setlength{\rightskip}{0pt plus 5cm}{\bf Vec2f} Triangle::p3}\hypertarget{class_triangle_a43b6750934c84aff2f34b38d2f26d012}{}\label{class_triangle_a43b6750934c84aff2f34b38d2f26d012}

\hypertarget{class_vec2f}{}\section{Vec2f Class Reference}
\label{class_vec2f}\index{Vec2f@{Vec2f}}


{\ttfamily \#include $<$vector2.\+h$>$}



Collaboration diagram for Vec2f\+:
% FIG 0
\subsection*{Public Member Functions}
\begin{DoxyCompactItemize}
\item 
\hyperlink{class_vec2f_a3582875fbf3badc6af02646e07bcf440}{Vec2f} ()
\item 
\hyperlink{class_vec2f_a3ce4919fdf7137dd1536b68f4aae53fb}{Vec2f} (double \+\_\+x, double \+\_\+y)
\item 
\hyperlink{class_vec2f_a3a41f568a8852d55f8d4989af776b798}{Vec2f} (const \hyperlink{class_vec2f}{Vec2f} \&v)
\item 
void \hyperlink{class_vec2f_a23721913973d2f6cbd337cf7355274e1}{set} (const \hyperlink{class_vec2f}{Vec2f} \&v)
\item 
double \hyperlink{class_vec2f_a9d8ee61baed27654281edb0bea4f022a}{dist2} (const \hyperlink{class_vec2f}{Vec2f} \&v) const 
\item 
double \hyperlink{class_vec2f_a648c71905bbd4376be6ffe729ab9a4e0}{dist} (const \hyperlink{class_vec2f}{Vec2f} \&v) const 
\item 
bool \hyperlink{class_vec2f_afb67dd49ed0ff76dbf4c4bd5344e2b59}{operator==} (const \hyperlink{class_vec2f}{Vec2f} \&right) const 
\item 
bool \hyperlink{class_vec2f_a32bddb3312af7d2158235c8bbb886324}{operator$<$} (const \hyperlink{class_vec2f}{Vec2f} \&right) const 
\end{DoxyCompactItemize}
\subsection*{Public Attributes}
\begin{DoxyCompactItemize}
\item 
double \hyperlink{class_vec2f_aae649f38fb692202ed76dd1783c02d1f}{x}
\item 
double \hyperlink{class_vec2f_a6215e122ad762df0beb90949f8e1859a}{y}
\end{DoxyCompactItemize}
\subsection*{Friends}
\begin{DoxyCompactItemize}
\item 
std\+::ostream \& \hyperlink{class_vec2f_a364e82f5d9f4720d6fc56554f4fbd69e}{operator$<$$<$} (std\+::ostream \&str, \hyperlink{class_vec2f}{Vec2f} p)
\end{DoxyCompactItemize}


\subsection{Constructor \& Destructor Documentation}
\index{Vec2f@{Vec2f}!Vec2f@{Vec2f}}
\index{Vec2f@{Vec2f}!Vec2f@{Vec2f}}
\subsubsection[{\texorpdfstring{Vec2f()}{Vec2f()}}]{\setlength{\rightskip}{0pt plus 5cm}Vec2f\+::\+Vec2f (
\begin{DoxyParamCaption}
{}
\end{DoxyParamCaption}
)\hspace{0.3cm}{\ttfamily [inline]}}\hypertarget{class_vec2f_a3582875fbf3badc6af02646e07bcf440}{}\label{class_vec2f_a3582875fbf3badc6af02646e07bcf440}
\index{Vec2f@{Vec2f}!Vec2f@{Vec2f}}
\index{Vec2f@{Vec2f}!Vec2f@{Vec2f}}
\subsubsection[{\texorpdfstring{Vec2f(double \+\_\+x, double \+\_\+y)}{Vec2f(double _x, double _y)}}]{\setlength{\rightskip}{0pt plus 5cm}Vec2f\+::\+Vec2f (
\begin{DoxyParamCaption}
\item[{double}]{\+\_\+x, }
\item[{double}]{\+\_\+y}
\end{DoxyParamCaption}
)\hspace{0.3cm}{\ttfamily [inline]}}\hypertarget{class_vec2f_a3ce4919fdf7137dd1536b68f4aae53fb}{}\label{class_vec2f_a3ce4919fdf7137dd1536b68f4aae53fb}
\index{Vec2f@{Vec2f}!Vec2f@{Vec2f}}
\index{Vec2f@{Vec2f}!Vec2f@{Vec2f}}
\subsubsection[{\texorpdfstring{Vec2f(const Vec2f \&v)}{Vec2f(const Vec2f &v)}}]{\setlength{\rightskip}{0pt plus 5cm}Vec2f\+::\+Vec2f (
\begin{DoxyParamCaption}
\item[{const {\bf Vec2f} \&}]{v}
\end{DoxyParamCaption}
)\hspace{0.3cm}{\ttfamily [inline]}}\hypertarget{class_vec2f_a3a41f568a8852d55f8d4989af776b798}{}\label{class_vec2f_a3a41f568a8852d55f8d4989af776b798}


\subsection{Member Function Documentation}
\index{Vec2f@{Vec2f}!dist@{dist}}
\index{dist@{dist}!Vec2f@{Vec2f}}
\subsubsection[{\texorpdfstring{dist(const Vec2f \&v) const }{dist(const Vec2f &v) const }}]{\setlength{\rightskip}{0pt plus 5cm}double Vec2f\+::dist (
\begin{DoxyParamCaption}
\item[{const {\bf Vec2f} \&}]{v}
\end{DoxyParamCaption}
) const\hspace{0.3cm}{\ttfamily [inline]}}\hypertarget{class_vec2f_a648c71905bbd4376be6ffe729ab9a4e0}{}\label{class_vec2f_a648c71905bbd4376be6ffe729ab9a4e0}


Here is the call graph for this function\+:
% FIG 1


\index{Vec2f@{Vec2f}!dist2@{dist2}}
\index{dist2@{dist2}!Vec2f@{Vec2f}}
\subsubsection[{\texorpdfstring{dist2(const Vec2f \&v) const }{dist2(const Vec2f &v) const }}]{\setlength{\rightskip}{0pt plus 5cm}double Vec2f\+::dist2 (
\begin{DoxyParamCaption}
\item[{const {\bf Vec2f} \&}]{v}
\end{DoxyParamCaption}
) const\hspace{0.3cm}{\ttfamily [inline]}}\hypertarget{class_vec2f_a9d8ee61baed27654281edb0bea4f022a}{}\label{class_vec2f_a9d8ee61baed27654281edb0bea4f022a}


Here is the caller graph for this function\+:
% FIG 2


\index{Vec2f@{Vec2f}!operator$<$@{operator$<$}}
\index{operator$<$@{operator$<$}!Vec2f@{Vec2f}}
\subsubsection[{\texorpdfstring{operator$<$(const Vec2f \&right) const }{operator<(const Vec2f &right) const }}]{\setlength{\rightskip}{0pt plus 5cm}bool Vec2f\+::operator$<$ (
\begin{DoxyParamCaption}
\item[{const {\bf Vec2f} \&}]{right}
\end{DoxyParamCaption}
) const\hspace{0.3cm}{\ttfamily [inline]}}\hypertarget{class_vec2f_a32bddb3312af7d2158235c8bbb886324}{}\label{class_vec2f_a32bddb3312af7d2158235c8bbb886324}
\index{Vec2f@{Vec2f}!operator==@{operator==}}
\index{operator==@{operator==}!Vec2f@{Vec2f}}
\subsubsection[{\texorpdfstring{operator==(const Vec2f \&right) const }{operator==(const Vec2f &right) const }}]{\setlength{\rightskip}{0pt plus 5cm}bool Vec2f\+::operator== (
\begin{DoxyParamCaption}
\item[{const {\bf Vec2f} \&}]{right}
\end{DoxyParamCaption}
) const\hspace{0.3cm}{\ttfamily [inline]}}\hypertarget{class_vec2f_afb67dd49ed0ff76dbf4c4bd5344e2b59}{}\label{class_vec2f_afb67dd49ed0ff76dbf4c4bd5344e2b59}
\index{Vec2f@{Vec2f}!set@{set}}
\index{set@{set}!Vec2f@{Vec2f}}
\subsubsection[{\texorpdfstring{set(const Vec2f \&v)}{set(const Vec2f &v)}}]{\setlength{\rightskip}{0pt plus 5cm}void Vec2f\+::set (
\begin{DoxyParamCaption}
\item[{const {\bf Vec2f} \&}]{v}
\end{DoxyParamCaption}
)\hspace{0.3cm}{\ttfamily [inline]}}\hypertarget{class_vec2f_a23721913973d2f6cbd337cf7355274e1}{}\label{class_vec2f_a23721913973d2f6cbd337cf7355274e1}


\subsection{Friends And Related Function Documentation}
\index{Vec2f@{Vec2f}!operator$<$$<$@{operator$<$$<$}}
\index{operator$<$$<$@{operator$<$$<$}!Vec2f@{Vec2f}}
\subsubsection[{\texorpdfstring{operator$<$$<$}{operator<<}}]{\setlength{\rightskip}{0pt plus 5cm}std\+::ostream\& operator$<$$<$ (
\begin{DoxyParamCaption}
\item[{std\+::ostream \&}]{str, }
\item[{{\bf Vec2f}}]{p}
\end{DoxyParamCaption}
)\hspace{0.3cm}{\ttfamily [friend]}}\hypertarget{class_vec2f_a364e82f5d9f4720d6fc56554f4fbd69e}{}\label{class_vec2f_a364e82f5d9f4720d6fc56554f4fbd69e}


\subsection{Member Data Documentation}
\index{Vec2f@{Vec2f}!x@{x}}
\index{x@{x}!Vec2f@{Vec2f}}
\subsubsection[{\texorpdfstring{x}{x}}]{\setlength{\rightskip}{0pt plus 5cm}double Vec2f\+::x}\hypertarget{class_vec2f_aae649f38fb692202ed76dd1783c02d1f}{}\label{class_vec2f_aae649f38fb692202ed76dd1783c02d1f}
\index{Vec2f@{Vec2f}!y@{y}}
\index{y@{y}!Vec2f@{Vec2f}}
\subsubsection[{\texorpdfstring{y}{y}}]{\setlength{\rightskip}{0pt plus 5cm}double Vec2f\+::y}\hypertarget{class_vec2f_a6215e122ad762df0beb90949f8e1859a}{}\label{class_vec2f_a6215e122ad762df0beb90949f8e1859a}

\hypertarget{classcmst_1_1_window}{}\section{cmst\+:\+:Window Class Reference}
\label{classcmst_1_1_window}\index{cmst\+::\+Window@{cmst\+::\+Window}}


{\ttfamily \#include $<$Window.\+h$>$}

\subsection*{Classes}
\begin{DoxyCompactItemize}
\item 
struct \hyperlink{structcmst_1_1_window_1_1_test}{Test}
\end{DoxyCompactItemize}
\subsection*{Public Member Functions}
\begin{DoxyCompactItemize}
\item 
\hyperlink{classcmst_1_1_graph2_d}{Graph2D} $\ast$ \hyperlink{classcmst_1_1_window_a83548b1c4406f37a812a0920fc4d6669}{cur\+Graph} ()\hypertarget{classcmst_1_1_window_a83548b1c4406f37a812a0920fc4d6669}{}\label{classcmst_1_1_window_a83548b1c4406f37a812a0920fc4d6669}

\begin{DoxyCompactList}\small\item\em Returns a pointer to the graph in display currently. \end{DoxyCompactList}\item 
void \hyperlink{classcmst_1_1_window_adfa83ea52f2c09def65b79cdf57092ad}{reset\+Cur\+Graph} (std\+::vector$<$ \hyperlink{classcmst_1_1_point2_d}{Point2D} $>$ \&points)
\item 
void \hyperlink{classcmst_1_1_window_a08365866ac2ffa9793c25c92750341e2}{reset\+Cur\+Graph} ()
\item 
void \hyperlink{classcmst_1_1_window_a83743944c1c6429f0eb9f3c72c9b7f22}{reset\+Cur\+Graph} (int n)
\item 
void \hyperlink{classcmst_1_1_window_aa4ad17303edd88d526b66f9e6918fe7a}{reset\+Cur\+Graph} (int low, int hi)
\item 
bool {\bfseries load} ()\hypertarget{classcmst_1_1_window_a3bb28914f7c58f3df9860ce333a8f25f}{}\label{classcmst_1_1_window_a3bb28914f7c58f3df9860ce333a8f25f}

\item 
void \hyperlink{classcmst_1_1_window_ae54c419d28bd352f52a3beffa6fc1f32}{reset\+Show\+Delaunay} ()\hypertarget{classcmst_1_1_window_ae54c419d28bd352f52a3beffa6fc1f32}{}\label{classcmst_1_1_window_ae54c419d28bd352f52a3beffa6fc1f32}

\begin{DoxyCompactList}\small\item\em Change whether the Delaunay diagram is to be drawn to the G\+L\+UT window. \end{DoxyCompactList}\item 
void \hyperlink{classcmst_1_1_window_a3f234daf3198e3611892515f1721de44}{reset\+Width} (int \hyperlink{classcmst_1_1_window_a5fc4ccbd9afed56cd17d341269028da2}{width})\hypertarget{classcmst_1_1_window_a3f234daf3198e3611892515f1721de44}{}\label{classcmst_1_1_window_a3f234daf3198e3611892515f1721de44}

\begin{DoxyCompactList}\small\item\em Record the width of current G\+L\+UT window. \end{DoxyCompactList}\item 
void \hyperlink{classcmst_1_1_window_a8c4d7788d1932e73397c20b7a9639d69}{reset\+Height} (int \hyperlink{classcmst_1_1_window_a6fc02b2afee52c0f71b6a3bd39c9210f}{height})\hypertarget{classcmst_1_1_window_a8c4d7788d1932e73397c20b7a9639d69}{}\label{classcmst_1_1_window_a8c4d7788d1932e73397c20b7a9639d69}

\begin{DoxyCompactList}\small\item\em Record the height of current G\+L\+UT window. \end{DoxyCompactList}\item 
int \hyperlink{classcmst_1_1_window_a5fc4ccbd9afed56cd17d341269028da2}{width} () const 
\item 
int \hyperlink{classcmst_1_1_window_a6fc02b2afee52c0f71b6a3bd39c9210f}{height} () const 
\item 
void \hyperlink{classcmst_1_1_window_a1d78ef796691e87ed3b4978f373c9890}{draw} ()
\item 
void \hyperlink{classcmst_1_1_window_a73eb23d7d7418cb288022e623e590461}{print\+Cur\+Info} ()
\item 
bool \hyperlink{classcmst_1_1_window_aa3f5edeebdd298190bdb4676b4838f75}{display\+Test} () const 
\item 
void \hyperlink{classcmst_1_1_window_acda99115d9c67f83de4d8f5a94f2647c}{generate\+Test} (int n)
\item 
void \hyperlink{classcmst_1_1_window_a471e1648f99754f7c2cbfbbb1e8ab556}{print\+Test\+Info} ()
\item 
int \hyperlink{classcmst_1_1_window_ae456c1bf45fc28133390fcc881fbd612}{test\+Display\+Num} () const 
\item 
void \hyperlink{classcmst_1_1_window_aa8dc3725888e12fc2c6e3626ab7b600a}{change\+Test\+Display} (int direc)
\item 
bool {\bfseries print\+To\+File} ()\hypertarget{classcmst_1_1_window_a1d2514395cd8e864ddcd6aadf127bbbc}{}\label{classcmst_1_1_window_a1d2514395cd8e864ddcd6aadf127bbbc}

\item 
void {\bfseries change\+M\+S\+T\+Display} (int direc)\hypertarget{classcmst_1_1_window_ae9d9bf19cad20b34a6ab850a8a218c13}{}\label{classcmst_1_1_window_ae9d9bf19cad20b34a6ab850a8a218c13}

\item 
void {\bfseries print\+S\+T\+Info} ()\hypertarget{classcmst_1_1_window_a5b39068b3053a7fe98f3d94de802fecb}{}\label{classcmst_1_1_window_a5b39068b3053a7fe98f3d94de802fecb}

\item 
void {\bfseries run\+Validate} ()\hypertarget{classcmst_1_1_window_ab2073132753d04d6cf7b882549f1ceea}{}\label{classcmst_1_1_window_ab2073132753d04d6cf7b882549f1ceea}

\end{DoxyCompactItemize}
\subsection*{Static Public Member Functions}
\begin{DoxyCompactItemize}
\item 
static \hyperlink{classcmst_1_1_window}{Window} $\ast$ \hyperlink{classcmst_1_1_window_a281790e82296e7be50c19520f136e345}{instance} ()
\end{DoxyCompactItemize}
\subsection*{Protected Attributes}
\begin{DoxyCompactItemize}
\item 
struct \hyperlink{structcmst_1_1_window_1_1_test}{cmst\+::\+Window\+::\+Test} \hyperlink{classcmst_1_1_window_aedae466fb2efd886cea6d775b20fabe3}{m\+\_\+test}\hypertarget{classcmst_1_1_window_aedae466fb2efd886cea6d775b20fabe3}{}\label{classcmst_1_1_window_aedae466fb2efd886cea6d775b20fabe3}

\begin{DoxyCompactList}\small\item\em The test. \end{DoxyCompactList}\end{DoxyCompactItemize}


\subsection{Detailed Description}
Manipulates the window.

Uses Singleton pattern. 

\subsection{Member Function Documentation}
\index{cmst\+::\+Window@{cmst\+::\+Window}!change\+Test\+Display@{change\+Test\+Display}}
\index{change\+Test\+Display@{change\+Test\+Display}!cmst\+::\+Window@{cmst\+::\+Window}}
\subsubsection[{\texorpdfstring{change\+Test\+Display(int direc)}{changeTestDisplay(int direc)}}]{\setlength{\rightskip}{0pt plus 5cm}void cmst\+::\+Window\+::change\+Test\+Display (
\begin{DoxyParamCaption}
\item[{int}]{direc}
\end{DoxyParamCaption}
)\hspace{0.3cm}{\ttfamily [inline]}}\hypertarget{classcmst_1_1_window_aa8dc3725888e12fc2c6e3626ab7b600a}{}\label{classcmst_1_1_window_aa8dc3725888e12fc2c6e3626ab7b600a}
If a test is being displayed, then changes the graph in the test that is being displayed.

If no test has been generated, does nothing. 
\begin{DoxyParams}{Parameters}
{\em direc} & If negative, display the last graph (if there is one); if positive, display the next graph (if there is one). \\
\hline
\end{DoxyParams}
\index{cmst\+::\+Window@{cmst\+::\+Window}!display\+Test@{display\+Test}}
\index{display\+Test@{display\+Test}!cmst\+::\+Window@{cmst\+::\+Window}}
\subsubsection[{\texorpdfstring{display\+Test() const }{displayTest() const }}]{\setlength{\rightskip}{0pt plus 5cm}bool cmst\+::\+Window\+::display\+Test (
\begin{DoxyParamCaption}
{}
\end{DoxyParamCaption}
) const\hspace{0.3cm}{\ttfamily [inline]}}\hypertarget{classcmst_1_1_window_aa3f5edeebdd298190bdb4676b4838f75}{}\label{classcmst_1_1_window_aa3f5edeebdd298190bdb4676b4838f75}
Returns if a test has been generated

\begin{DoxyReturn}{Returns}
If a test has been generated 
\end{DoxyReturn}
\index{cmst\+::\+Window@{cmst\+::\+Window}!draw@{draw}}
\index{draw@{draw}!cmst\+::\+Window@{cmst\+::\+Window}}
\subsubsection[{\texorpdfstring{draw()}{draw()}}]{\setlength{\rightskip}{0pt plus 5cm}void cmst\+::\+Window\+::draw (
\begin{DoxyParamCaption}
{}
\end{DoxyParamCaption}
)}\hypertarget{classcmst_1_1_window_a1d78ef796691e87ed3b4978f373c9890}{}\label{classcmst_1_1_window_a1d78ef796691e87ed3b4978f373c9890}
Draws the current graph


\begin{DoxyItemize}
\item Points\+: definitely
\item Delaunay Diagram\+: change whether to draw it by \hyperlink{classcmst_1_1_window_ae54c419d28bd352f52a3beffa6fc1f32}{Window\+::reset\+Show\+Delaunay()}
\item M\+ST\+: definitely
\item Other spanning trees\+: draws one of them 
\end{DoxyItemize}\index{cmst\+::\+Window@{cmst\+::\+Window}!generate\+Test@{generate\+Test}}
\index{generate\+Test@{generate\+Test}!cmst\+::\+Window@{cmst\+::\+Window}}
\subsubsection[{\texorpdfstring{generate\+Test(int n)}{generateTest(int n)}}]{\setlength{\rightskip}{0pt plus 5cm}void cmst\+::\+Window\+::generate\+Test (
\begin{DoxyParamCaption}
\item[{int}]{n}
\end{DoxyParamCaption}
)}\hypertarget{classcmst_1_1_window_acda99115d9c67f83de4d8f5a94f2647c}{}\label{classcmst_1_1_window_acda99115d9c67f83de4d8f5a94f2647c}
Generates a test of n graphs and display the first one.


\begin{DoxyParams}{Parameters}
{\em n} & The number of graphs in the test to be generated \\
\hline
\end{DoxyParams}
\index{cmst\+::\+Window@{cmst\+::\+Window}!height@{height}}
\index{height@{height}!cmst\+::\+Window@{cmst\+::\+Window}}
\subsubsection[{\texorpdfstring{height() const }{height() const }}]{\setlength{\rightskip}{0pt plus 5cm}int cmst\+::\+Window\+::height (
\begin{DoxyParamCaption}
{}
\end{DoxyParamCaption}
) const\hspace{0.3cm}{\ttfamily [inline]}}\hypertarget{classcmst_1_1_window_a6fc02b2afee52c0f71b6a3bd39c9210f}{}\label{classcmst_1_1_window_a6fc02b2afee52c0f71b6a3bd39c9210f}
Return the height of current G\+L\+UT window.

\begin{DoxyReturn}{Returns}
The height of current G\+L\+UT window 
\end{DoxyReturn}
\index{cmst\+::\+Window@{cmst\+::\+Window}!instance@{instance}}
\index{instance@{instance}!cmst\+::\+Window@{cmst\+::\+Window}}
\subsubsection[{\texorpdfstring{instance()}{instance()}}]{\setlength{\rightskip}{0pt plus 5cm}static {\bf Window}$\ast$ cmst\+::\+Window\+::instance (
\begin{DoxyParamCaption}
{}
\end{DoxyParamCaption}
)\hspace{0.3cm}{\ttfamily [inline]}, {\ttfamily [static]}}\hypertarget{classcmst_1_1_window_a281790e82296e7be50c19520f136e345}{}\label{classcmst_1_1_window_a281790e82296e7be50c19520f136e345}
Return the pointer to the instance of \hyperlink{classcmst_1_1_window}{cmst\+::\+Window} class.

\begin{DoxyReturn}{Returns}
the pointer to the instance 
\end{DoxyReturn}
\index{cmst\+::\+Window@{cmst\+::\+Window}!print\+Cur\+Info@{print\+Cur\+Info}}
\index{print\+Cur\+Info@{print\+Cur\+Info}!cmst\+::\+Window@{cmst\+::\+Window}}
\subsubsection[{\texorpdfstring{print\+Cur\+Info()}{printCurInfo()}}]{\setlength{\rightskip}{0pt plus 5cm}void cmst\+::\+Window\+::print\+Cur\+Info (
\begin{DoxyParamCaption}
{}
\end{DoxyParamCaption}
)}\hypertarget{classcmst_1_1_window_a73eb23d7d7418cb288022e623e590461}{}\label{classcmst_1_1_window_a73eb23d7d7418cb288022e623e590461}
Prints information about the current displayed graph to console

Information including numbers and computational time \index{cmst\+::\+Window@{cmst\+::\+Window}!print\+Test\+Info@{print\+Test\+Info}}
\index{print\+Test\+Info@{print\+Test\+Info}!cmst\+::\+Window@{cmst\+::\+Window}}
\subsubsection[{\texorpdfstring{print\+Test\+Info()}{printTestInfo()}}]{\setlength{\rightskip}{0pt plus 5cm}void cmst\+::\+Window\+::print\+Test\+Info (
\begin{DoxyParamCaption}
{}
\end{DoxyParamCaption}
)}\hypertarget{classcmst_1_1_window_a471e1648f99754f7c2cbfbbb1e8ab556}{}\label{classcmst_1_1_window_a471e1648f99754f7c2cbfbbb1e8ab556}
Prints information about the test that has been generated to console.

If no test has been generated, then nothing is printed. \index{cmst\+::\+Window@{cmst\+::\+Window}!reset\+Cur\+Graph@{reset\+Cur\+Graph}}
\index{reset\+Cur\+Graph@{reset\+Cur\+Graph}!cmst\+::\+Window@{cmst\+::\+Window}}
\subsubsection[{\texorpdfstring{reset\+Cur\+Graph(std\+::vector$<$ Point2\+D $>$ \&points)}{resetCurGraph(std::vector< Point2D > &points)}}]{\setlength{\rightskip}{0pt plus 5cm}void cmst\+::\+Window\+::reset\+Cur\+Graph (
\begin{DoxyParamCaption}
\item[{std\+::vector$<$ {\bf Point2D} $>$ \&}]{points}
\end{DoxyParamCaption}
)}\hypertarget{classcmst_1_1_window_adfa83ea52f2c09def65b79cdf57092ad}{}\label{classcmst_1_1_window_adfa83ea52f2c09def65b79cdf57092ad}
Reset the current graph with a vector of points. 
\begin{DoxyParams}{Parameters}
{\em points} & A vector of points. \\
\hline
\end{DoxyParams}
\index{cmst\+::\+Window@{cmst\+::\+Window}!reset\+Cur\+Graph@{reset\+Cur\+Graph}}
\index{reset\+Cur\+Graph@{reset\+Cur\+Graph}!cmst\+::\+Window@{cmst\+::\+Window}}
\subsubsection[{\texorpdfstring{reset\+Cur\+Graph()}{resetCurGraph()}}]{\setlength{\rightskip}{0pt plus 5cm}void cmst\+::\+Window\+::reset\+Cur\+Graph (
\begin{DoxyParamCaption}
{}
\end{DoxyParamCaption}
)}\hypertarget{classcmst_1_1_window_a08365866ac2ffa9793c25c92750341e2}{}\label{classcmst_1_1_window_a08365866ac2ffa9793c25c92750341e2}
Reset the current graph with cmst\+::\+Testcase\+Generator

The size of the graph is defaulted. \index{cmst\+::\+Window@{cmst\+::\+Window}!reset\+Cur\+Graph@{reset\+Cur\+Graph}}
\index{reset\+Cur\+Graph@{reset\+Cur\+Graph}!cmst\+::\+Window@{cmst\+::\+Window}}
\subsubsection[{\texorpdfstring{reset\+Cur\+Graph(int n)}{resetCurGraph(int n)}}]{\setlength{\rightskip}{0pt plus 5cm}void cmst\+::\+Window\+::reset\+Cur\+Graph (
\begin{DoxyParamCaption}
\item[{int}]{n}
\end{DoxyParamCaption}
)}\hypertarget{classcmst_1_1_window_a83743944c1c6429f0eb9f3c72c9b7f22}{}\label{classcmst_1_1_window_a83743944c1c6429f0eb9f3c72c9b7f22}
Reset the current graph with n random generated points.


\begin{DoxyParams}{Parameters}
{\em n} & The size of the graph to be generated. \\
\hline
\end{DoxyParams}
\index{cmst\+::\+Window@{cmst\+::\+Window}!reset\+Cur\+Graph@{reset\+Cur\+Graph}}
\index{reset\+Cur\+Graph@{reset\+Cur\+Graph}!cmst\+::\+Window@{cmst\+::\+Window}}
\subsubsection[{\texorpdfstring{reset\+Cur\+Graph(int low, int hi)}{resetCurGraph(int low, int hi)}}]{\setlength{\rightskip}{0pt plus 5cm}void cmst\+::\+Window\+::reset\+Cur\+Graph (
\begin{DoxyParamCaption}
\item[{int}]{low, }
\item[{int}]{hi}
\end{DoxyParamCaption}
)}\hypertarget{classcmst_1_1_window_aa4ad17303edd88d526b66f9e6918fe7a}{}\label{classcmst_1_1_window_aa4ad17303edd88d526b66f9e6918fe7a}
Reset the current graph with random generated points.

The size of the graph to be generated is randomly selected between low and hi. 
\begin{DoxyParams}{Parameters}
{\em low} & The least number of points to be generated. \\
\hline
{\em hi} & The most number of points to be generated. \\
\hline
\end{DoxyParams}
\index{cmst\+::\+Window@{cmst\+::\+Window}!test\+Display\+Num@{test\+Display\+Num}}
\index{test\+Display\+Num@{test\+Display\+Num}!cmst\+::\+Window@{cmst\+::\+Window}}
\subsubsection[{\texorpdfstring{test\+Display\+Num() const }{testDisplayNum() const }}]{\setlength{\rightskip}{0pt plus 5cm}int cmst\+::\+Window\+::test\+Display\+Num (
\begin{DoxyParamCaption}
{}
\end{DoxyParamCaption}
) const\hspace{0.3cm}{\ttfamily [inline]}}\hypertarget{classcmst_1_1_window_ae456c1bf45fc28133390fcc881fbd612}{}\label{classcmst_1_1_window_ae456c1bf45fc28133390fcc881fbd612}
Returns the number of graphs in the test that has been generated. \begin{DoxyReturn}{Returns}
the number of graphs in the test that has been generated. 
\end{DoxyReturn}

\begin{DoxyRetVals}{Return values}
{\em 0} & If no test has been generated. \\
\hline
\end{DoxyRetVals}
\index{cmst\+::\+Window@{cmst\+::\+Window}!width@{width}}
\index{width@{width}!cmst\+::\+Window@{cmst\+::\+Window}}
\subsubsection[{\texorpdfstring{width() const }{width() const }}]{\setlength{\rightskip}{0pt plus 5cm}int cmst\+::\+Window\+::width (
\begin{DoxyParamCaption}
{}
\end{DoxyParamCaption}
) const\hspace{0.3cm}{\ttfamily [inline]}}\hypertarget{classcmst_1_1_window_a5fc4ccbd9afed56cd17d341269028da2}{}\label{classcmst_1_1_window_a5fc4ccbd9afed56cd17d341269028da2}
Return the width of current G\+L\+UT window.

\begin{DoxyReturn}{Returns}
The width of current G\+L\+UT window 
\end{DoxyReturn}


The documentation for this class was generated from the following files\+:\begin{DoxyCompactItemize}
\item 
Window.\+h\item 
Window.\+cpp\end{DoxyCompactItemize}

\clearpage


\begin{thebibliography}{9}

\bibitem{example}
This is an example.

\end{thebibliography}


\end{document}
